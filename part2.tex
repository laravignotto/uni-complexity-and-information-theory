\part{Complessità}
% LEZIONE 9


\chapter{Introduzione}

Libro di Papadimitriou main reference.

In questa parte utilizzeremo come modello di computazione le \textbf{macchine di Turing} (TM). Esistono diversi modelli di TM: macchine di Turing multinastro, macchine di Turing input/output, macchine di Turing con oracolo, macchine di Turing nondeterministiche. Le TM verranno utilizzate per confrontare i diversi risultati di complessità che possiamo ottenere.

Ci concentreremo sia su \textbf{complessità temporale} (time complexity) che \textbf{spaziale} (space complexity). Il focus non sarà sulla complessità di un dato algoritmo, ma sulla complessità di un problema. I \textbf{problemi} possono essere classificati come di decisione (decision problems), di funzione (function problems), o di ottimizzazione (optimization problems).
\begin{itemize}
    \item \textbf{Decision problem} $P:\text{inputs}\to\{yes,no\}$
    \item \textbf{Function problem} computare una data funzione, ad esempio l'ordinamento di una lista
    \item \textbf{Optimization problem} tra tutti i possibili output, si vuole trovare quello che minimizza o massimizza una funzione di costo. 
\end{itemize}

\paragraph{Esempio} Sia $G=(V,E)$ un grafo, e $u,v\in V$ due nodi. 
\begin{itemize}
    \item[--] decidere se esiste un cammino da $u$ a $v$ è un problema di decisione
    \item[--] trovare un cammino da $u$ a $v$ è un problema di funzione
    \item[--] trovare il cammino più corto da $u$ a $v$ è un problema di ottimizzazione
\end{itemize}

In questo corso ci concentreremo sui problemi di decisione. Se si ha una soluzione per un problema di funzione o di ottimizzazione, si possiede automaticamente una soluzione per il problema di decisione.\medskip

% (disegno nelle note)
Immaginiamo tutti gli input possibili al problema dell'esempio precedente come ad un insieme infinito di tuple $(G,u,v)$. Questo insieme si può dividere in due: il sottoinsieme dei \emph{yes} di tutte le codifiche binarie di triple $(G,u,v)$ tali che esiste un cammino in $G$ da $u$ a $v$, e, inversamente, il sottoinsieme \emph{no}.
\begin{center}
    \begin{tikzpicture}
        \node[ellipse,
            draw,
            minimum width=6cm,
            minimum height=3cm] (A) at (0,0) { };
        \draw (0,-1.5) -- (0,1.5);
        \node[above left] at (A.north west) {$L$};
        \node[below] at (A.220) {\emph{yes}};
        \node[below] at (A.-40) {\emph{no}};

        \node at (-2,0.4) {$\bullet$};
        \node at (-0.8,0.2) {$\bullet$};
        \node at (-1.2,-0.4) {$\bullet$};
        \node at (-0.8,-0.8) {$\dots$};
        \node at (1.6,-0.2) {$\bullet$};
        \node at (1.2,0.4) {$\bullet$};
        \node at (0.8,-0.8) {$\dots$};

        \draw[->,decorate,decoration=snake] (-2,0.4) -- (-3.6,1);
        \node[left] at (-3.6,1) {$(G,u,v)$};
    \end{tikzpicture}
\end{center}
La codifica binaria di una tripla è una stringa del tipo $1011\dots$. Più precisamente, è una stringa sull'alfabeto $\Sigma=\{0,1\}$. L'insieme di tutte le possibili stringhe binarie è $\Sigma^*$. Questo insieme è quindi il linguaggio $L$ sottoinsieme di $\Sigma^*$, ovvero $L\subseteq\Sigma^*$.
$$
    L = \{ \bin(G,u,v)|ln~G~u\to v \}
$$

\paragraph{Esempio} Consideriamo interi rappresentati in binario. Vogliamo decidere se un dato intero $x$ è divisibile per 4.
$$
    \bin(x) = 10\dots11
$$
in questo caso non è divisibile per 4. Un numero binario è divisibile per 4 se e solo se i due bit meno significativi sono 0.
$$
    \bin(x) = x_n,x_{n-1},\dots,x_1,x_0 \quad \Leftrightarrow \quad x_0 = 0 \text{ and } x_1 = 0
$$
Il linguaggio indicato da questo problema di decisione è
$$
    L = \{ x\in\{0,1\}^* | x=x_n,x_{n-1},\dots,x_1,x_0 \land x_0 = 0 \land x_1 = 0 \}
$$


\paragraph{Esempio: palindromo} Decidere se una stringa è palindroma, con $\Sigma = \{0,1\}$.
$$
    x_1,x_2,x_3,\dots,x_3,x_2,x_1
$$
Ad esempio, $x=101$ è palindroma, mentre $x=1010$ non lo è. Cerchiamo il linguaggio $L=\{x|x \text{ è palindroma}\}$. Utilizziamo una macchina di Turing.
\begin{center}
    \begin{tikzpicture}
        % Draw the horizontal strip
        \draw (0,0) -- (5,0);
        \draw (0,.5) -- (5,.5);
        %  Draw the squares
        \foreach \x in {0.5,1,...,4.5} {
          \draw (\x,0) -- (\x,.5);
        }

        % Label the squares
        % \foreach \x in {2.25,2.75,3.25,4.25,4.75,5.25} {
        %   \node[above] at (\x,0) {$*$};
        % }
        \node[above] at (0.75,0) {$\rhd$};
        \node[above] at (1.25,0) {$x_1$};
        \node[above] at (1.75,0) {$x_2$};
        \node[above] at (3.75,0) {$x_n$};
        \node[above] at (2.8,0.1) {\Huge \dots};
        \node[above] at (4.25,0) {$\sqcup$};
        \draw [decorate,
            decoration = {brace,mirror,amplitude=5pt}] (1,-0.1) -- (4,-0.1);
        \node[below] at (2.5,-0.2) {$x$};
        \node[below] at (0.75,0) {\LARGE$\substack{\uparrow\\s}$};
        % \node at (0.75,-0.65) {$s$};
    \end{tikzpicture}
\end{center}
Si parte dallo stato $s$ e si vuole finire nello stato $p$ solo quando $x$ è palindroma. Per decidere se $x$ è palindroma, si può leggere $x_1$, ricordarne il valore nello stato del puntatore, e poi confrontarlo con $x_n$. Se sono uguali, si ripete lo stesso procedimento con $x_2$ e $x_{n-1}$, e così via. Se si arriva a $x_n$ e $x_1$ senza aver trovato una discrepanza, allora $x$ è palindroma. Se invece si trova una discrepanza, allora $x$ non è palindroma. Le transizioni sono le seguenti:
\begin{align*}
    \delta(s,\rhd) &= (q,\rhd,\to)\\
    \delta(q,1) &= (q_1,\rhd,\to)\\
    \delta(q,0) &= (q_0,\rhd,\to)\\
\end{align*}
\textcolor{Red}{TODO: finire di scrivere le transizioni}
Questa macchina eseguirà un numero quadratico di passi per controllare se la stringa $x$ è palinfroma: $O(|x|^2)$.\medskip

Se si vuole controllare in C (o in un altro linguaggio) se una stringa è palindroma, si può scrivere un programma che confronta il primo e l'ultimo carattere, poi il secondo e il penultimo, e così via, eseguendo un numero lineare di passi. La complessità è $O(|x|)$. Questo è un esempio di come la complessità di un problema dipenda dal modello di computazione utilizzato.


\section{Tesi di Church-Turing Estesa} La tesi di Church-Turing afferma che ogni cosa che può essere computata, può essere computata da una macchina di Turing.

La versione estesa afferma che tutti i modelli (ragionevoli) di calcolo sono correlati polinomialmente. Questo significa che se un problema è risolvibile in tempo polinomiale in un modello di computazione, allora è risolvibile in tempo polinomiale in ogni modello di computazione.

In altre parole, la tesi di Church-Turing estesa afferma che la complessità computazionale di un problema è indipendente dal modello di calcolo utilizzato per risolverlo.
$$
    \underset{\text{problema}}{P} \to \text{TMs }O(f(n)) \to \underset{\text{modello}}{M}~O(p(f(n)))  
$$
Ma è vera anche la direzione contraria. \textcolor{Red}{TODO: ???}




%%%%%%%%%%%%%%%%%%%%%%%%%%%%
\chapter{Macchine di Turing e Unlimited Register Machines}


\section{Definizioni}

\begin{definition}[Configurazione]
    Una configurazione è una tripla $(q,w,u)$, con
    \begin{itemize}
        \item $q\in K\cup\{\text{yes, no, halt}\}$
        \item $w,u\in\Sigma^*$
    \end{itemize}
\end{definition}
Ad esempio, graficamente, una configurazione è
\begin{center}
    \begin{tikzpicture}
        % Draw the horizontal strip
        \draw (0,0) -- (6,0);
        \draw (0,.5) -- (6,.5);
        %  Draw the squares
        \foreach \x in {0.5,1,1.5,2,3,3.5,4.5,5,5.5} {
          \draw (\x,0) -- (\x,.5);
        }

        \node[above] at (0.75,0) {$\rhd$};
        \node[above] at (1.25,0) {$x_1$};
        \node[above] at (1.75,0) {$x_2$};
        \node[above] at (2.5,0) {\dots};
        \node[above] at (3.25,0) {$x_i$};
        \node[above] at (4,0) {\dots};
        \node[above] at (4.75,0) {$x_n$};
        \node[above] at (5.25,0) {$\sqcup$};
        \draw [decorate,
            decoration = {brace,amplitude=5pt}] (0.5,0.6) -- (3.48,0.6);
        \node[above] at (2,0.7) {$w$};
        \draw [decorate,
            decoration = {brace,amplitude=5pt}] (3.52,0.6) -- (5,0.6);
        \node[above] at (4.25,0.7) {$u$};
        \node[below] at (3.25,0) {\LARGE$\substack{\uparrow\\q}$};
    \end{tikzpicture}
\end{center}


\begin{definition}[Configurazione Iniziale]
    La configurazione iniziale su una stringa $x$ è una tripla 
    $$
    (1,\rhd,x)
    $$
\end{definition}

\begin{definition}[Configurazioni Finali]
    Le configurazioni finali su una stringa $x$ sono una tripla 
    $$
    (H,w,u)
    $$
    dove $H\in\{\text{yes, no, halt}\}$.
\end{definition}


\begin{definition}[Passo di Computazione]
    $$
        (q,w,u) \overset{\delta}{\to} (q',w',u') 
    $$
\end{definition}
Ad esempio, il passo di computazione è $(s,\rhd,001)\to(q,\rhd 0,01)$
\begin{center}
    \begin{tikzpicture}
        % Draw the horizontal strip
        \draw (0,0) -- (4,0);
        \draw (0,.5) -- (4,.5);
        %  Draw the squares
        \foreach \x in {0.5,1,...,3.5} {
          \draw (\x,0) -- (\x,.5);
        };
        \node[above] at (0.75,0) {$\rhd$};
        \node[above] at (1.25,0) {$0$};
        \node[above] at (1.75,0) {$0$};
        \node[above] at (2.25,0) {$1$};
        \node[above] at (2.75,0) {$\sqcup$};
        \node[above] at (3.25,0) {$\sqcup$};
        \node[below] at (0.75,0) {\LARGE\cancel{$\substack{\uparrow\\s}$}};
        \node[below] at (1.25,0) {\LARGE$\substack{\uparrow\\q}$};
    \end{tikzpicture}
\end{center}
Eseguito applicando $\delta(s,\rhd)=(q,\rhd,\to)$.


\begin{definition}[Time Complexity per una MdT $\bm{\mathcal{M}}$ sull'input $\bm{x}$]
    $\mathcal{M}$ ha time complexity $t$ su $x$ se dopo esattamente $t$ passi si raggiunge una configurazione finale.
    $$
        (s,\rhd,x)\underbrace{\to\dots\to}_{t\text{ passi}}(H,w,u)
    $$
    Indicata in breve con $(s,\rhd,x)\to^t(H,w,u)$.\medskip

    \noindent $\mathcal{M}$ ha time complexity $f:\mathbb{N}\to\mathbb{N}$ se, $\forall x\in\Sigma^*$, $(s,\rhd,x)\to^t(H,w,u)$ con $t\leq f(|x|)$.
\end{definition}
La dimensione dell'input (bit length dell'input) è $|x|$. Questa è una complessità nel caso peggiore ($\leq$). Non stiamo utilizzando la notazione big-O.



\section{Unlimited Register Machines}
Una Unlimited Register Machine (URM) è una macchina di Turing con un numero illimitato di registri. 
\begin{table}[H]
    \centering
    \begin{tabular}{c|c|}
    \cline{2-2}
    $R_0$ & $r_0$ \\ \cline{2-2} 
    $R_1$ & $r_1$ \\ \cline{2-2} 
          & \dots \\ \cline{2-2} 
    $R_m$ & $r_m$ \\ \cline{2-2} 
          & \dots 
    \end{tabular}
\end{table}
Ogni registro contiene un numero naturale. Quindi, il contenuto del registro $R_m$ sarà $r_m\in\mathbb{N}$. Le operazioni possibili sono:
\begin{itemize}
    \item \textbf{incremento} $S(i)$: $r_i:=r_i+1$
    \item \textbf{azzeramento} $Z(i)$: $r_i:=0$
    \item \textbf{trasferimento} $T(i,j)$: $r_j:=r_i$, ovvero trasferisco il contenuto del registro $R_i$ nel registro $R_j$
    \item \textbf{jump} $J(i,j,k)$: se $r_i=r_j$ allora salta all'istruzione $k$, altrimenti prosegue con l'istruzione successiva
\end{itemize}

% Macchina con un unbounded number of registers.

% All'inizio, l'input $x$ è in $R_0$. Abbiamo una specie di counter.

\paragraph{Esempio} Dati $x,y\in\mathbb{N}$, decidere se $x=y$.

\subparagraph{Con una MdT} Si può utilizzare una macchina di Turing che contiene la rappresentazione binaria dei due interi, separati da un separatore.
\begin{center}
    \begin{tikzpicture}
        % Draw the horizontal strip
        \draw (0,0) -- (6.5,0);
        \draw (0,.5) -- (6.5,.5);
        %  Draw the squares
        \foreach \x in {0.5,1,1.5,2.5,3,3.5,4,5,5.5,6} {
          \draw (\x,0) -- (\x,.5);
        };
        \node[above] at (0.75,0) {$\rhd$};
        \node[above] at (1.25,0) {$x_1$};
        \node[above] at (2,0) {\dots};
        \node[above] at (2.75,0) {$x_n$};
        \node[above] at (3.25,0) {$\sqcup$};
        \node[above] at (3.75,0) {$y_1$};
        \node[above] at (4.5,0) {\dots};
        \node[above] at (5.25,0) {$y_m$};
        \node[above] at (5.75,0) {$\sqcup$};
        \node[below] at (0.75,0) {\LARGE$\substack{\uparrow\\s}$};
        \draw [decorate,
            decoration = {brace,amplitude=5pt}] (1,0.6) -- (3,0.6);
        \node[above] at (2,0.7) {$x$};
        \draw [decorate,
            decoration = {brace,amplitude=5pt}] (3.5,0.6) -- (5.5,0.6);
        \node[above] at (4.5,0.7) {$y$};
    \end{tikzpicture}
\end{center}
Questa macchina richiede, nel caso peggiore, un numero quadratico di passi per terminare. La complessità è $\Theta(|x|^2)$.

\subparagraph{Con una URM} Possiamo utilizzare una URM con $x$ e $y$ rispettivamente nei registri $R_0$ e $R_1$.
\begin{table}[H]
    \centering
    \begin{tabular}{c|c|}
    \cline{2-2}
    $R_0$ & $x$ \\ \cline{2-2} 
    $R_1$ & $y$ \\ \cline{2-2} 
          &    
    \end{tabular}
\end{table}
\noindent Alla fine, scriveremo 1 in $R_0$ se $x=y$, 0 altrimenti. Le istruzioni sono le seguenti:
\begin{enumerate}
    \item $J(0,1,4)$
    \item $Z(0)$
    \item $J(0,0,100)$
    \item $Z(0)$
    \item $S(0)$
\end{enumerate}
In questo caso, la complessità si può calcolare in due modi.

\begin{definition}[Time Complexity su URM]~
    \begin{itemize}
        \item \emph{Uniform cost criterium} (criterio del costo uniforme): numero di istruzioni eseguite.
        \item \emph{Logarithmic cost criterium} (criterio del costo logaritmico): ogni istruzione ha un costro proporzionale al numero di cifre coinvolte.
    \end{itemize}
\end{definition}
Quindi, per questa macchina, la complessità è 
\begin{itemize}
    \item utilizzando il criterio del costo uniforme: $\Theta(1)$
    \item utilizzando il criterio del costo logaritmico: $\Theta(|x|+|y|)$
\end{itemize}
Nel secondo caso, ci si avvicina al costo per la macchina di Turing. 

Mentre le macchine di Turing sono un modello di computazione sequenziale, nelle unlimited register machines si ha l'istruzione \emph{jump}.
