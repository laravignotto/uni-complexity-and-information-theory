\documentclass[a4paper,11pt]{book}

\usepackage[italian]{babel}
% \usepackage[english]{babel}
\usepackage[utf8]{inputenc}
\usepackage{alltt}
\usepackage[T1]{fontenc}
\usepackage{anyfontsize}
\usepackage{amsmath}
\DeclareMathOperator*{\bin}{bin}
\usepackage{mathtools}
\usepackage{bm}
\usepackage{makeidx}
\usepackage[italian]{varioref}
% \usepackage[english]{varioref}
\usepackage{wasysym}
\usepackage{amssymb}
\usepackage{pifont}
\usepackage{listings}
% \usepackage{mathdots}
% \DeclareFontFamily{U}{dice3d}{}
% \DeclareFontShape{U}{dice3d}{m}{n}{<-> s*[4] dice3d}{}
% \usepackage{epsdice}
\usepackage[top=2.2cm,bottom=1.6cm,left=2.4cm,right=1.6cm]{geometry}
% \usepackage[margin=2cm]{geometry}
\usepackage[font=small,labelfont=bf]{caption}
\usepackage{subcaption}
\usepackage{amsthm}
\theoremstyle{definition}
\usepackage{fancyvrb}
\usepackage[dvipsnames,table,xcdraw]{xcolor}
\usepackage{mdframed}
\usepackage{hyperref}
\usepackage{thmtools}
\usepackage{enumitem,multicol}
\usepackage{algorithm}
\usepackage[noend]{algpseudocode}
\usepackage{nicefrac}
\usepackage{enumitem}
\usepackage{cancel}
\usepackage{tikz}
\usetikzlibrary{arrows.meta,calc,decorations.markings,math,arrows.meta}
\usetikzlibrary{positioning}
\usetikzlibrary{trees}
\usetikzlibrary{shapes.geometric}
\usetikzlibrary{decorations.pathmorphing}
\tikzset{snake it/.style={decorate, decoration=snake}}
\usetikzlibrary{tikzmark}
\usepackage{eurosym}
\usepackage{booktabs}
% \usepackage{multirow}
\usepackage[pages=some,placement=top]{background}

% \usepackage{amsthm}
\theoremstyle{definition}

\newmdtheoremenv[backgroundcolor=Plum!5]{theorem}{\color{Plum}Teorema}[section]
\newmdtheoremenv[backgroundcolor=Mahogany!5]{corollary}{\color{Mahogany}Corollario}[theorem]
\newmdtheoremenv[backgroundcolor=JungleGreen!5]{definition}{\color{JungleGreen}Definizione}[section]
\newmdtheoremenv[backgroundcolor=OrangeRed!5]{lemma}[theorem]{\color{OrangeRed}Lemma}
\newmdtheoremenv[backgroundcolor=DarkOrchid!5]{property}{\color{DarkOrchid}Proprietà}[section]
\newmdtheoremenv[backgroundcolor=Orange!5]{proposition}{\color{Orange}Proposizione}[section]

\usepackage{pgfplots}
% \pgfmathdeclarefunction{poiss}{1}{%
%   \pgfmathparse{(#1^x)*exp(-#1)/(x!)}%
% }

\newlength{\mytextsize}
\makeatletter
      \setlength{\mytextsize}{\f@size pt}
\makeatother

\newcommand{\JoinUp}[5]{\begin{tikzpicture}[remember picture,overlay,line width=0.05\mytextsize]
    \draw([shift={(#1\mytextsize,#2\mytextsize)}]pic cs:start#5) -- ++(0pt,0.7\mytextsize) -| ([shift={(#3\mytextsize,#4\mytextsize)}]pic cs:end#5);
    \node at (6.5,1.8) {$\neq$};
    \end{tikzpicture}}
\newcommand{\JoinDown}[5]{\begin{tikzpicture}[remember picture,overlay,line width=0.05\mytextsize]
    \draw([shift={(#1\mytextsize,#2\mytextsize)}]pic cs:start#5) -- ++(0pt,-0.7\mytextsize) -| ([shift={(#3\mytextsize,#4\mytextsize)}]pic cs:end#5);
    \node at (9.4,.15) {$\neq$};
    \end{tikzpicture}}

\def\rdots{\rotatebox[origin=l]{29}{$\scriptscriptstyle\ldots\mathstrut$}}

\allowdisplaybreaks

\graphicspath{{./figures/}}

% \makeatletter
% \newcommand{\chapterkey}[1]{%
%   \renewcommand{\chapter@key}{#1}%
% }
% \newcommand{\chapter@key}{???} % initialize
% \renewcommand{\thechapter}{\chapter@key}
% \makeatother

% \renewcommand*{\listtheoremname}{Lista di Teoremi e Definizioni}

% \setlength{\parindent}{2.2cm}

%\newcommand{\vin}{\rotatebox[origin=c]{-90}{$\in$}}


% better circled numbers
% \newcommand*\circled[1]{\tikz[baseline=(char.base)]{
%            \node[shape=circle,draw,inner sep=2pt] (char) {#1};}}

% \setcounter{tocdepth}{4}
\setcounter{tocdepth}{3}
\setcounter{secnumdepth}{3}

\makeindex

   
%%%%%%%%%%
\title{\fontfamily{cmss}\fontsize{50}{60}\selectfont \textbf{Complessità e Teoria dell'Informazione}}
\author{\Large \fontfamily{cmss}\selectfont Prof.ssa~Carla Piazza}
\date{\fontfamily{cmss}\selectfont Appunti di Claudio Desideri e Lara Vignotto -- a.a. 2023/2024}


%%%%%%%%%%
\begin{document}

\backgroundsetup{
scale=.185,
angle=0,
opacity=0.4,  %% adjust
contents={\includegraphics[keepaspectratio]{pexels-gül-işık-2128249}}
}
\BgThispage

\maketitle

\vfill
% \begin{multicols}{2}
% {
% \tiny
% \scriptsize
% \footnotesize
% \begin{quote}
\tableofcontents
% \end{quote}
% }
% \end{multicols}

\newpage


%%%%%%%%%% 
\chapter{Introduction}

Programma:
\begin{itemize}
    \item Teoria dell'Informazione
    \item Teoria della Complessità
    \item Algoritmi su Grafi ecc \dots
\end{itemize}





%%%%%%%%%%%%%%%%%%
\chapter{Teoria dell'Informazione}

\section{Entropia}
% LEZIONE 1
Claude Shannon, 1948, \textit{A Mathematical Theory of Communication}.

Un messaggio è una sequenza di lettere (simboli) da un alfabeto. Qual è l'informazione in una frase (messaggio)? Come possiamo misurare la quantità di informazione? Dipende dal contesto.

\subparagraph{Esempio} Il messaggio è ``Piove!''. Qual è la quantità di informazione? Per il signor Muller, che vive a Vienna, dove piove spesso, la quantità di informazione è bassa. Per Fatima, che vive nel deserto, invece, è alta.\bigskip

Si ha quindi che
\begin{itemize}
    \item Bassa probabilità di un evento 
\end{itemize}

...

% LEZIONE 2
\section{Teorema di Bayes}
Sappiamo che nel caso di due \textbf{eventi indipendenti}, la probabilità congiunta è
$$
    p(a_i,b_j) = p(a_i)\cdot p(b_j)
$$
Nel caso, invece, di due \textbf{eventi dipendenti} si ha
$$
    p(a_i,b_j) = p(a_i|b_j)\cdot p(b_j) = p(b_j|a_i)\cdot p(a_i)
$$
e quindi
$$
p(a_i|b_j) = \dfrac{p(b_j|a_i)\cdot p(a_i)}{p(b_j)}
$$
con $p(b_j)$ fattore di normalizzazione.

\textcolor{Red}{TODO: vedere al capitolo 2 del libro il prior, poterior, likelihood, ecc.}



\section{Proprietà dell'Entropia}
Libro, pag.~33. Le proprietà della funzione di entropia sono:
\begin{itemize}
    \item $H(P)\geq 0$ con uguaglianza iff $p_i=1$ per un qualche $i$. In particolare, $H(P)=0$ iff $\exists i ~p_i=1$, ovvero quando c'è un evento certo l'entropia è nulla.
    \item L'entropia è massimizzata se la distribuzione $p$ è uniforme.
\end{itemize}
Analizziamo meglio e dimostriamo la seconda proprietà.

\begin{property}[Entropia massima] 
    $$
        \mathcal{H}(P)\leq \log_2|P|
    $$
    con $|P|$ numero di eventi ($|X|$), e
    $$
        \mathcal{H}\left(\frac{1}{k},\frac{1}{k},\dots,\frac{1}{k}\right) = \log_2 k
    $$
    dove $k$ è il numero di eventi, ovvero $|P|$.
\end{property}

\subparagraph{Dimostrazione} La dimostrazione è basata su proprietà di funzioni complesse, in particolare sulla diseguaglianza di Jensen, la quale è una disuguaglianza che lega il valore di una funzione convessa al valore della medesima funzione calcolata nel valor medio del suo argomento.

Sia $\bm{f(x)=-x\log_2x}$ (cfr.~definizione di entropia). Vogliamo controllare se è concava o convessa, calcoliamo quindi la sua derivata seconda:
$$
    f''(x) = -\frac{1}{x}
$$
Sappiamo che $0\leq x\leq 1$ perché è una probabilità, e quindi abbiamo che $-\nicefrac{1}{x}<0$: la funzione è con\-ca\-va. Prendiamo ora due punti $x_1$ e $x_2$, e un punto $x$ tra i due.
% $0\leq-\nicefrac{1}{x}\leq 1$ perché è una probabilità, e quindi abbiamo che $x<0$: la funzione è concava. Prendiamo ora due punti $x_1$ e $x_2$, e un punto $x$ tra i due.

\textcolor{Red}{TODO: disegno}

Abbiamo che la media pesata di $x_1$ e $x_2$ è
$$
    x = \lambda x_1 + (1-\lambda)x_2 \qquad \text{con} \quad 0\leq\lambda\leq 1
$$
con $\lambda$ il peso. In particolare, se $\lambda=1$ allora $x=x_1$, se $\lambda=0$ allora $x=x_2$, e se $\lambda=\nicefrac{1}{2}$ allora $x$ si troverà esattamente a metà tra $x_1$ e $x_2$. Se prendiamo la combinazione lineare di $f(x_1)$ e $f(x_2)$, otteniamo la seguente di\-su\-gua\-glian\-za:
$$
    \lambda f(x_1) + (1-\lambda)f(x_2) \leq f(\lambda x_1 + (1-\lambda)x_2)
$$
Tale disuguaglianza si può generalizzare alla combinazione lineare di un qualsiasi numero di punti. Se $f''(x)\leq 0$, $\forall x\in[x_1,x_n]$ (ovvero $f''(x)$ è concava) si ha la disuguaglianza di Jensen: 
$$
\textcolor{ForestGreen}{\sum_{i=1}^n\lambda_if(x_i)} \leq \textcolor{Cerulean}{f\left(\sum_{i=1}^n\lambda_ix_i\right)}
$$
con $\lambda_i\geq0$ e $\sum_{i=1}^n\lambda_i=1$. La disuguaglianza cambia verso ($\geq$) quando la funzione è convessa, cioè $f''(x)\geq 0$. Ricordiamo che vogliamo arrivare alla combinazione lineare dove i punti hanno la stessa probabilità. Quindi con
$$
    \lambda_i=\frac{1}{k} \qquad\qquad |P|=|X|=k \qquad\qquad P=\{p_1,\dots,p_k\}
$$
scriviamo la disuguaglianza di Jensen come
\begin{eqnarray*}
    \textcolor{ForestGreen}{\sum_{i=1}^k\underbrace{\frac{1}{k}}_{\lambda_i}\underbrace{(-p_i\log_2p_i)}_{f(x_i)}} & \leq & \underbrace{\textcolor{Cerulean}{-\left(\sum_{i=1}^k\frac{1}{k}p_i\right)\log_2\left(\sum_{i=1}^k\frac{1}{k}p_i\right)}}_{f(\sum\lambda_ix_i)}\\
    -\cancelto{\text{\footnotesize semplifichiamo perché $k>0$}}{\frac{1}{k}}\sum_{i=1}^kp_i\log_2p_i & \leq & -\cancel{\frac{1}{k}}\log_2\frac{1}{k}\\
    \mathcal{H}(P) & \leq & \log_2k=\mathcal{H}\left(\frac{1}{k},\dots,\frac{1}{k}\right)
\end{eqnarray*}
Ovvero l'entropia è massima per la distribuzione uniforme. \hfill$\Box$\bigskip 

\begin{property}[Entropia Congiunta]
    Siano $P,Q$ due distribuzioni, e $x_i,y_j$ coppia di eventi tali che $x_i\in P$ e $y_j\in Q$. L'entropia congiunta di $P,Q$ è:
    $$
        \mathcal{H}(P,Q) = -\sum_{i,j}p(x_i,y_j)\log_2(p(x_i,y_j))
    $$
    Se $x$ e $y$ sono indipendenti (quindi la probabilità congiunta è il prodotto delle due probabilità) l'entropia è additiva:
    $$
        \mathcal{H}(P,Q) = \mathcal{H}(P) + \mathcal{H}(Q)
        % ESERCIZIO ESAME
    $$
\end{property}
La somma è possibile perché usiamo i logaritmi, e una delle loro proprietà è $\log(a\cdot b)=\log(a)+\log(b)$.


\subsection{Scomponibilità dell'Entropia}
Libro, pag.~33. Sia $P$ una distribuzione (vettore) di probabilità, e $X$ delle variabili.
\begin{eqnarray*}
    P &=& \{ p_1,~p_2,~...,~p_n\}\\
    X &=& \{ \underbrace{x_1}_{p_1},\underbrace{x_2,...,x_n}_{1-p_1}\}
\end{eqnarray*}
In questo contesto, la probabilità, ad esempio, del secondo evento $x_2$ è, normalizzata, pari a $\nicefrac{p_2}{1-p_1}$, e quella dell'ultimo elemento $x_n$ è $\nicefrac{p_n}{1-p_1}$.

\subparagraph{Esempio} Abbiamo una moneta regolare. Al primo lancio esce $H$, e come risultati desiderati per il secondo e terzo lancio vogliamo $T$ e $T$. Abbiamo quindi:
\begin{eqnarray*}
    \underbrace{H}_{\substack{p_1=\frac{1}{2}\\\frac{1}{2}}}~\underbrace{T\quad T}_{\substack{1-p_1=\frac{1}{2}\\\frac{1}{4}\quad\frac{1}{4}}}
\end{eqnarray*}

La quantità di informazione ricevuta da $P$ è uguale a quella ricevuta dal processo in due passaggi.
\begin{eqnarray*}
    \mathcal{H}(P)  & = & \sum p_i\log_2\frac{1}{p_i}\\
                    & = & \underbrace{\mathcal{H}(p_1,1-p_1)+(1-p_1)\cdot\mathcal{H}\underbrace{\left(\frac{p_2}{1-p_1},\frac{p_3}{1-p_1},\dots,\frac{p_n}{1-p_1}\right)}_{\substack{\text{$p_i$ normalizzati la cui somma è 1}\\1-p_1=p_2+p_3+\dots+p_n}}}_{\text{si possono dividere in diversi punti, ottenendo, ad esempio, tante entropie}}
\end{eqnarray*}

\textcolor{Red}{TODO: vdere proprietà libro pag 33 sez 2.6?}


\section{Inferenza}
\textcolor{Red}{TODO: capitolo 3, esercizio 3.8 pag 57}

\section{Compressione}
\textcolor{Red}{TODO: capitolo 4, esercizio 4.1 pag 66}


\section{Il Teorema Della Codifica Sorgente}
Studiamo $\mathcal{H}(\{p,1-p\})$, con $0\leq p\leq 1$
\begin{eqnarray*}
    \mathcal{H}(\{p,1-p\})  & = &\\
                            & = & p\log\frac{1}{p} + (1-p)\log\frac{1}{1-p}\\
                            & = & -p\log(p) - (1-p)\log(1-p)\\
                            & = & \mathcal{H}(p)
\end{eqnarray*}

\textcolor{Red}{TODO: finire}


\textcolor{Red}{LEZ 3}

\textcolor{Red}{TODO: soluzione esercizio 4.1}

\textcolor{Red}{TODO: muddy children puzzle}



\section{Codici Simbolo}

\begin{definition}[Alfabeti di input/output]
    \begin{align*}
        \text{Alfabeto di input} \quad &\mathcal{A}=\{a_1,a_2,\dots,a_k\}\\
        \text{Alfabeto di output} \quad &\mathcal{B}=\{b_1,b_2,\dots,b_D\}\\
    \end{align*}
\end{definition}

\begin{definition}[Codice]
    Sia $\mathcal{A}^*$ un messaggio (sequenza di caratteri) sull'alfabeto $\mathcal{A}$. Il \textbf{codice} $c$ è
    $$
        c:\mathcal{A}^*\to\mathcal{B}^*
    $$
    ovvero un messaggio dall'alfabeto $\mathcal{A}$ all'alfabeto $\mathcal{B}$ (iniettiva).
\end{definition}
Con $\mathcal{A}^*=\bigcup_{n\in\mathbb{N}}A^n$, ovvero l'insieme di tutte le possibili stringhe che si possono creare utilizzando l'alfabeto $\mathcal{A}$, compresa la stringa vuota.\medskip 

Si vuole comprimere il messaggio in modo da ottenere il messaggio più corto possibile. Per farlo, utilizziamo una codifica.
\begin{definition}[Codifica]
    Una codifica è una funzione 
    $$
        \varphi:\mathcal{A}\to\mathcal{B}^*
    $$ 
    Inoltre
    $$
        \varphi(\underbrace{x_1,x_2,\dots,x_m}_{\in\mathcal{A}^*}) = \varphi(x_1)\varphi(x_2)\dots\varphi(x_m)
    $$
\end{definition}

\paragraph{Esempio 1} Alfabeti: $\mathcal{A}=\{a,b,c\}$, $\mathcal{B}=\{0,1\}$; codifica: $\varphi(a)=0$, $\varphi(b)=10$, $\varphi(c)=01$. È una buona codifica? No, perché è ambigua. Ad esempio
$$
    \varphi(ab) = 010 = \varphi(ca)
$$
È iniettiva nella codifica ma non sul messaggio. Una codifica di questo tipo viene detta \textbf{not uniquely decodable} (non univocamente decodificabile).

\begin{definition}[Univocamente decodificabile]
    Un codice $\varphi:\mathcal{A}\to\mathcal{B}^*$ è univocamente decodificabile (uniquely decodable) se
    $$
        \forall m_1,m_2\in\mathcal{A}^* \qquad \varphi(m_1)\neq\varphi(m_2)
    $$
\end{definition}
In questo corso non utilizzeremo codici non univocamente decodificabili.

\paragraph{Esempio 2} Alfabeti: $\mathcal{A}=\{a,b,c\}$, $\mathcal{B}=\{0,1\}$; codifica: $\varphi(a)=0$, $\varphi(b)=01$, $\varphi(c)=011$. Ad esempio, il messaggio $aabcbbca$ viene codificato come $\varphi(aabcbbca)=000101101010110$. È univocamente decodificabile (UD)?

Sì, è UD con delay 1. Ad ogni 0, si controlla il carattere successivo: se è un altro 0, la lettera è $a$, altrimenti si prosegue fino al primo 1 per decidere se è $b$ o $c$. Il delay 1 è riferito allo zero che si incontra, che significa l'inizio di un'altra lettera.

\paragraph{Esempio 3} Alfabeti: $\mathcal{A}=\{a,b,c\}$, $\mathcal{B}=\{0,1\}$; codifica: $\varphi(a)=00$, $\varphi(b)=1$, $\varphi(c)=10$. Ad esempio, il messaggio $bcbaca$ viene codificato come $\varphi(bcbaca)=1101001000$. È UD? 

Sì, si controlla se c'è un numero pari o dispari di 0 dopo un 1: se è pari, si tratta di una $b$ seguita da una o più $a$, se è dispari di una $c$, eventualmente seguita da una o più $a$. È UD con unbounded delay.\bigskip 

Abbiamo visto che nel caso di distribuzione uniforme, la quantità di informazione è pari all'entropia. Quanto è complesso computare una codifica/decodifica?

Poiché l'alfabeto di input è binario, per rappresentare una codifica si può utilizzare un albero binario. Quello per l'\textbf{Esempio 2} è il seguente:
\begin{center}
    \begin{tikzpicture}[grow=right]
        \node {$\bullet$}
        child {
            node {$\bullet$}        
            edge from parent 
            node[below] {$1$}
        }
        child {
            node {$a$}        
            child {
                    node {$b$}
                    child {
                        node {$c$}
                        edge from parent
                        node[below] {$1$}
                    }
                    child {
                        node {$\bullet$}
                        edge from parent
                        node[above] {$0$}
                    }
                    edge from parent
                    node[below] {$1$}
                }
                child {
                    node {$\bullet$}
                    edge from parent
                    node[above] {$0$}
                }
            edge from parent         
            node[above] {$0$}
        };
    \end{tikzpicture}
\end{center}
Quello per l'\textbf{Esempio 3} è il seguente: \textcolor{Red}{TODO: fixare in modo che gli archi ad $a$ e $c$ siano inclinati}
\begin{center}
    \begin{tikzpicture}[grow=right]
        \node {$\bullet$}
        child {
            node {$b$}
            child {
                node {$c$}        
                edge from parent 
                node[below] {$0$}
            }        
            edge from parent 
            node[below] {$1$}
        }
        child {
            node {$\bullet$} 
            child {
                node {$a$}
                edge from parent
                node[above] {$0$}
            }
            edge from parent         
            node[above] {$0$}
        };
    \end{tikzpicture}
\end{center}
Quando si finisce in un nodo con un'etichetta ci si deve chiedere se la conclusione è che ci si può fermare. Il grado (fattore) di diramazione è pari alla cardinalità dell'alfabeto di output. Inoltre, se l'albero ha altezza $h$, tutte le codifiche hanno lunghezza $h$.

Ci chiediamo, qual è una codifica sicuramente UD e senza delay?

\begin{definition}[Codice prefisso]
    \begin{eqnarray*}
        &\varphi : \mathcal{A} \to \mathcal{B}^* \text{ è un codice prefisso}&\\
        &\Updownarrow&\\
        &\forall a_i,a_j\in\mathcal{A} \quad \varphi(a_i)\in\mathcal{B}^*\text{ non è un prefisso di }\varphi(a_j)\in\mathcal{B}^*&
    \end{eqnarray*}
\end{definition}
Al posto di codice prefisso (prefix code) utilizzeremo il termine prefisso (prefix). \textcolor{Red}{è corretto?}

\paragraph{Esempio 3 (vedi sopra)} $\varphi$ non è un prefisso, perché la codifica di $b$ è $1$, che è un prefisso della codifica di $c$, ovvero 10.

\begin{lemma}
    $\varphi$ è un codice prefisso \quad $\Rightarrow$ \quad $\varphi$ è UD senza delay
\end{lemma}
Per memorizzare l'albero si utilizza CONSTANT SPACE, che è un sottoinsieme di LINEAR TIME. Ciò equivale a dire che è possibile computarlo con un automa.

\paragraph{Esempio 3 (vedi sopra)} $\varphi(bbb)=111$: il messaggio di input ha lunghezza 3, e viene codificato in un messaggio di lunghezza uguale (3). 

$\varphi(aca)=001000$: il messaggio di input ha lunghezza 3, e viene codificato in un messaggio di lunghezza 6.

Il numero di possibili messaggi di lunghezza 3 in output è $2^3=8$.

\begin{definition}[Lunghezza media di una codifica, EL (Expected Length)]
    Siano $\varphi:\mathcal{A}\to\mathcal{B}^*$ una codifica, e $P$ una distribuzione di probabilità sull'alfabeto di input $\mathcal{A}$
    $$
        EL(\varphi) = \sum_{i=1}^n p(a_i)\cdot|\varphi(a_i)|
    $$
    con $|\varphi(a_i)|=l_i$ lunghezza della codifica di $a_i$.
\end{definition}

\paragraph{Esempio} Supponiamo che nell'\textbf{Esempio 3} le probabilità siano $p(a)=\nicefrac{1}{2}$, $p(b)=\nicefrac{1}{4}$, $p(c)=\nicefrac{1}{4}$. La lunghezza media è
$$
    EL(\varphi) = \frac{1}{2}\cdot 2 + \frac{1}{4}\cdot 1 + \frac{1}{4}\cdot 2 = 1+\frac{3}{4}
$$
Immaginiamo una codifica $\varphi^*$ diversa, per la quale $|\varphi^*(a)|=1$, $|\varphi^*(b)|=2$, $|\varphi^*(c)|=2$. La lunghezza media è
$$
    EL(\varphi^*) = \frac{1}{2}\cdot 1 + \frac{1}{4}\cdot 2 + \frac{1}{4}\cdot 2 = \frac{3}{2}
$$
Abbiamo che $EL(\varphi)>EL(\varphi^*)$, quindi $\varphi^*$ è migliore di $\varphi$.\bigskip

Si vuole trovare la codifica con la minor EL sotto l'assunzione che la sorgente del messaggio non abbia memoria. Dati $\mathcal{A}$, $\mathcal{B}^*$, $P$, si vuole trovare la miglior codifica $\varphi$. Non è sufficiente considerare solo codici prefix-free.

Possiamo raggiungere il minimo di EL considerando i codici prefisso. EL è codificata da $\mathcal{H}(P)$. I codici possono essere asintotticamente ottimali, o ottimali.


\section{Limite imposto dalla Decodificabilità Univoca}

È possibile definire un codice $\varphi:\mathcal{A}\to\mathcal{B}^*$ che sia UD, date le lunghezze delle codifiche $l_1,\dots,l_k$? Definiamo della terminologia:
$$
    k=|\mathcal{A}| \qquad \qquad D=|\mathcal{B}|(=2\text{ nel libro})
$$

\begin{theorem}[Disuguagianza di Kraft-McMillan, o Teorema Inverso]
    $$
        \varphi \text{ è UD}
        \quad \Rightarrow \quad
        \sum_{i=1}^k \frac{1}{D^{l_i}} \leq 1 ~ = ~ \sum_{i=1}^k D^{-l_i} \leq 1
    $$
\end{theorem}
Se $>1$ non esiste un codice UD con tale lunghezza.

\paragraph{Esempio} $\mathcal{A}=\{a,b,c\}$, $\mathcal{B}=\{0,1\}$, $l_1=1$, $l_2=1$, $l_3=2$ (lunghezze delle codifiche di $a,b,c$). Applicando il teorema si ottiene
$$
    \frac{1}{2^1}+\frac{1}{2^1}+\frac{1}{2^2} > 1
$$
Infatti, non importa dove si sceglie di codificare la $c$, la codifica non è UD.
\begin{center}
    \begin{tikzpicture}[grow=right]
        \tikzstyle{level 2}=[sibling distance=.9cm]
        \node {$\bullet$}
        child {
            node {$b$}
            child {
                [dashed] node[right] {$c$}
                edge from parent
            }
            child {
                [dashed] node[right] {$c$}
                edge from parent
            }    
            edge from parent
        }
        child {
            node {$a$}
            child {
                [dashed] node[right] {$c$}
                edge from parent
            }
            child {
                [dashed] node[right] {$c$}
                edge from parent
            }
            edge from parent
        };
    \end{tikzpicture}
\end{center}

\paragraph{Esempio} $\mathcal{A}=\{a,b,c\}$, $\mathcal{B}=\{0,1\}$, $l_1=1$, $l_2=2$, $l_3=2$. Applicando il teorema si ottiene
$$
    \frac{1}{2^1}+\frac{1}{2^2}+\frac{1}{2^2} = 1
$$
è UD (cfr.~anche Teorema Diretto).
\begin{center}
    \begin{tikzpicture}[grow=right]
        \tikzstyle{level 2}=[sibling distance=.9cm]
        \node {$\bullet$}
        child {
            node {$\bullet$}
            child {
                node {$c$}
                edge from parent
            }
            child {
                node {$b$}
                edge from parent
            }    
            edge from parent
        }
        child {
            node {$a$}
            child {
                node {$\bullet\}$}
                edge from parent
            }
            child {
                node {$\bullet\}$}
                edge from parent
            }
            edge from parent
        };
    \node[text width=6cm,align=left] at (7.5,.8) {non conviene mettere $b$ o $c$ qui, perché non sarebbero prefix};
    \draw[->] (4.3,.9) node {} -- (3.3,1.2) [dashed] node {};
    \draw[->] (4.3,.6) node {} -- (3.3,.3) [dashed] node {};
    \end{tikzpicture}
\end{center}

\begin{theorem}[Teorema Diretto]
    $$
        \sum_{i=1}^k D^{-l_i} \leq 1
        \quad \Rightarrow \quad
        \exists \varphi \text{ prefisso con lunghezze } l_1,\dots,l_k
    $$
\end{theorem}
Prefisso è più forte di UD.\medskip 

I due risultati (teoremi) affermano che, anche se ci limitiamo all'utilizzo di codici prefisso, non perdiamo alcuna potenza nella compressione. È sufficiente l'utilizzo dei codici prefisso, comprimono a sufficienza.

\paragraph{Dimostrazione Teorema Inverso, caso prefisso + Teorema Diretto} Sia $\varphi$ un codice prefisso e $l_1\leq l_2\leq\dots\leq l_k=l$. Consideriamo l'albero $D$-ario (ogni nodo ha $D$ figli) che rappresenta $\varphi$, di altezza $l$.
\begin{center}
    \begin{tikzpicture}[grow=right]
        \tikzstyle{level 2}=[sibling distance=.5cm]
        \node {$\bullet$}
        child {
            node {$\bullet$}
            child {
                [dotted] node[right] {$\dots\bullet$}
                edge from parent
            }
            child {
                [dotted] node[right] {$\dots\bullet$}
                edge from parent
            }
            child {
                [dotted] node[right] {$\dots\bullet$}
                edge from parent
            }    
            edge from parent
        }
        child {
            node {$\bullet$}
            child {
                [dotted] node[right] {$\dots\bullet$}
                edge from parent
            }
            child {
                [dotted] node[right] {$\dots\bullet$}
                edge from parent
            }
            child {
                [dotted] node[right] {$\dots\bullet$}
                edge from parent
            }
            edge from parent
        }
        child {
            node {$\bullet$}
            child {
                [dotted] node[right] {$\dots\bullet$}
                edge from parent
            }
            child {
                [dotted] node[right] {$\dots\bullet$}
                edge from parent
            }
            child {
                [dotted] node[right] {$\dots\bullet$}
                edge from parent
            }
            edge from parent
        };
    \draw[<->] (4,-2.5) node {} -- (0,-2.5) node {};
    \node[below] at (2,-2.5) {$l$};
    \end{tikzpicture}
\end{center}

La codifica più lunga $a_k$ è in una delle foglie. Supponiamo che la codifica $a_i$ sia in uno dei nodi interni. Nessuno dei nodi interni (e in particolare nessuna delle foglie) del sottoalbero con $a_i$ come radice può essere utilizzato per un'altra codifica.
\begin{center}
    \begin{tikzpicture}[grow=right]
        \tikzstyle{level 1}=[level distance=2cm,sibling distance=2.4cm]
        \tikzstyle{level 2}=[level distance=2cm,sibling distance=.8cm]
        \node {$\bullet$}
        child {
            node {$\bullet$}
            child {
                [dotted] node[right] {$\qquad\dots\qquad\bullet$}
                edge from parent
            }
            child {
                [dotted] node[right] {$\qquad\dots\qquad\bullet$}
                edge from parent
            }
            child {
                [dotted] node[right] {$\qquad\dots\qquad\bullet$}
                edge from parent
            }    
            edge from parent
        }
        child {
            node {$\bullet$}
            child {
                [dotted] node[right] {$\qquad\dots\qquad\bullet$}
                edge from parent
            }
            child {
                [dotted] node[right] {$\qquad\dots\qquad a_k$}
                edge from parent
            }
            child {
                [dotted] node[right] {$\qquad\dots\qquad\bullet$}
                edge from parent
            }
            edge from parent
        }
        child {
            node {$\bullet$}
            child {
                [dotted] node[right] {$a_i$}
                edge from parent
            }
            child {
                [dotted] node[right] {$\qquad\dots\qquad\bullet$}
                edge from parent
            }
            child {
                [dotted] node[right] {$\qquad\dots\qquad\bullet$}
                edge from parent
            }
            edge from parent
        };
        \node[isosceles triangle,
            isosceles triangle apex angle=30,
            draw,
            rotate=180,
            fill=Gray!30,
            minimum size=1cm] (T)at (6.1,1.65){};
        \node at (5.8,1.65) {\tiny VIETATO};
    \end{tikzpicture}
\end{center}
In un tale albero, il numero di foglie è pari a $D^l$. 
\begin{center}
    \begin{tikzpicture}[grow=right]
        \tikzstyle{level 2}=[sibling distance=.5cm]
        \node {$\bullet$}
        child {
            node {$\bullet$}
            child {
                [dotted] node[right] {$\dots\bullet$}
                edge from parent
            }
            child {
                [dotted] node[right] {$\dots\bullet$}
                edge from parent
            }
            child {
                [dotted] node[right] {$\dots\bullet$}
                edge from parent
            }    
            edge from parent
        }
        child {
            node {$\bullet$}
            child {
                [dotted] node[right] {$\dots\bullet$}
                edge from parent
            }
            child {
                [dotted] node[right] {$\dots\bullet$}
                edge from parent
            }
            child {
                [dotted] node[right] {$\dots\bullet$}
                edge from parent
            }
            edge from parent
        }
        child {
            node {$\bullet$}
            child {
                [dotted] node[right] {$\dots\bullet$}
                edge from parent
            }
            child {
                [dotted] node[right] {$\dots\bullet$}
                edge from parent
            }
            child {
                [dotted] node[right] {$\dots\bullet$}
                edge from parent
            }
            edge from parent
        };
    \draw[<->] (4.5,-2.1) node {} -- (4.5,2.1) node {};
    \node[right] at (4.5,0) {$D^l$};
    \end{tikzpicture}
\end{center}
Il numero di foglie di un sottoalbero che parte da un nodo interno $a_i$ è $D^{l-l_i}$.
\begin{center}
\begin{tikzpicture}
    \node[isosceles triangle,
        isosceles triangle apex angle=40,
        draw,
        rotate=180,
        fill=Gray!30,
        minimum size=2.5cm] (T)at (0,0){};
    \draw[<->] (-2.55,-1.5) node {} -- (.85,-1.5) node {};
    \node[below] at (-0.85,-1.5) {$l-l_i$};
    \node at (-2.55,0) {$\bullet$};
    \node[left] at (-2.55,0) {$a_i$};
    \node[right] at (.85,1) {$\bullet$};
    \node[right] at (.85,0.1) {$\vdots$};
    \node[right] at (.85,-1) {$\bullet$};
    \draw[<->] (1.6,1.25) node {} -- (1.6,-1.25) node {};
    \node[right] at (1.6,0) {$D^{l-l_i}$};
\end{tikzpicture}
\end{center}
Quindi, riassumendo:
\begin{center}
    \begin{tikzpicture}[grow=right]
        \tikzstyle{level 1}=[level distance=2cm,sibling distance=2.4cm]
        \tikzstyle{level 2}=[level distance=2cm,sibling distance=.8cm]
        \node {$\bullet$}
        child {
            node {$\bullet$}
            child {
                [dotted] node[right] {$\qquad\dots\qquad\bullet$}
                edge from parent
            }
            child {
                [dotted] node[right] {$\qquad\dots\qquad\bullet$}
                edge from parent
            }
            child {
                [dotted] node[right] {$\qquad\dots\qquad\bullet$}
                edge from parent
            }    
            edge from parent
        }
        child {
            node {$\bullet$}
            child {
                [dotted] node[right] {$\qquad\dots\qquad\bullet$}
                edge from parent
            }
            child {
                [dotted] node[right] {$\qquad\dots\qquad a_k$}
                edge from parent
            }
            child {
                [dotted] node[right] {$\qquad\dots\qquad\bullet$}
                edge from parent
            }
            edge from parent
        }
        child {
            node {$\bullet$}
            child {
                [dotted] node[right] {$a_i$}
                edge from parent
            }
            child {
                [dotted] node[right] {$\qquad\dots\qquad\bullet$}
                edge from parent
            }
            child {
                [dotted] node[right] {$\qquad\dots\qquad\bullet$}
                edge from parent
            }
            edge from parent
        };
        \node[isosceles triangle,
            isosceles triangle apex angle=30,
            draw,
            rotate=180,
            fill=Gray!30,
            minimum size=1cm] (T)at (6.1,1.65){};
        \draw[<->] (-.1,-4) node {} -- (6.4,-4) node {};
        \node[below] at (3.15,-4) {$l$};
        \draw[<->] (6.8,1.15) node {} -- (6.8,2.15) node {};
        \node[right] at (6.8,1.65) {$D^{l-l_i}$};
        \draw[<->] (8.2,-3.3) node {} -- (8.2,3.3) node {};
        \node[right] at (8.2,0) {$D^l$};
    \end{tikzpicture}
\end{center}
Scriviamo che la differenza tra il numero totale di foglie $D^l$ e il numero di foglie dei sottoalberi (ovvero il numero di foglie vietate a causa dei sottoalberi creati da ogni $a_i$) è maggiore o uguale a 1:
\begin{align*}
    D^l - \sum_{i=1}^{k-1} D^{l-l_i} & \geq 1\\
    D^l\left(1-\sum_{i=1}^{k-1} D^{-l_i}\right) & \geq 1\\
    1 - \sum_{i=1}^{k-1} D^{-l_i} & \geq D^{-l_k}\\
    1 & \geq \sum_{i=1}^k D^{-l_i} %\qquad \square
\end{align*}
\hfill $\square$\medskip

Per il \textbf{Teorema Diretto}, si può leggere la dimostrazione ``al contrario''. In altre parole, vengono date le istruzioni per costruire l'albero, ovvero si disegna l'albero completo e si inizia ad etichettare. Più precisamente, si prende $a_1$, lo si mette a lunghezza $l_1$, e fino alle foglie si segna il resto dell'albero (il sottoalbero con radice $a_1$) come vietato. Si continua così per tutti gli $a_i$. \hfill $\square$

\paragraph{Dimostrazione Teorema Inverso, caso $\bm{\varphi}$ UD} Siano $l_1\leq\dots\leq l_k=l$ le lunghezze delle codifiche di $a_1,\dots,a_k$. Consideriamo $\mathcal{N}(n,h)$ numero di stringhe su $\mathcal{A}^n$ (ovvero di lunghezza $n$) che hanno una codifica $\varphi$ UD di lunghezza $h$. Sia $|\mathcal{B}^h|=D^h$ (numero di stringhe di lunghezza $h$ su $\mathcal{B}$). Poiché $\varphi$ è UD, $\mathcal{N}(n,h)\leq D^h$.
$$
    \sum_{i=1}^k D^{-l_i} = D^{-l_1} + D^{-l_2} + \dots + D^{-l_k} 
$$

Studiamo la crescita di tale oggetto alla potenza di $n$, quando $n$ va ad infinito. Questo perché, se la somma è $>1$, allora la potenza va ad infinito; se la somma è $<1$, allora la potenza va a 0 (è limitata); se la somma è $=1$, allora la potenza va a 1.
$$
    \forall n \qquad \left(D^{-l_1} + D^{-l_2} + \dots + D^{-l_k}\right)^n \qquad (\text{chiamiamola }\alpha^n)
$$

Se si svolge l'elevamento a potenza di tale polinomio, si otterrà una serie di addendi del seguente tipo:
$$
    D^{-l_1\cdot n} +
    \dots +
    D^{(-l_1)+(-l_2)+(-l_1)+\dots} +
    \dots +
    D^{-l_k\cdot n}
$$
dove $-l_1\cdot n$ è la lunghezza della codifica della stringa $a_1a_1\dots a_1$, con $n$ ripetizioni di $a_1$, ovvero $|\varphi(a_1\dots a_1)|=l_1\cdot n$. Allo stesso modo, $-l_k\cdot n$ è la lunghezza della codifica della stringa $a_ka_k\dots a_k$, con $n$ ripetizioni di $a_k$. Un generico esponente all'interno, ad esempio $-(l_1+l_2+l_1+\dots)$ è una somma di lunghezze di codifiche, e ammonta alla lunghezza di una generica stringa. Ad esempio, $-(10)$, se chiamo $10=h$.

Si ha che $D^{-h}$ si verifica nella somma esattamente $\mathcal{N}(n,h)$ volte (alcune delle quali saranno 0). Quindi, la somma $D^{-l_1\cdot n}+\dots+D^{-l_k\cdot n}$ si può riscrivere come 
\begin{align*}
    \mathcal{N}(n,0)D^{-0} + \mathcal{N}(n,1)D^{-1} + \dots + \underbrace{\mathcal{N}(n,n\cdot l)}_{\geq 1}D^{-n\cdot l} & \quad\leq\quad
    D^0D^{-0} + D^1D^{-1} + \dots + D^{n\cdot l}D^{-n\cdot l}\\
    1+1+\dots+1 & \quad\leq\quad n\cdot l + 1\\
    \forall n \qquad \alpha^n &\quad\leq\quad \underbrace{l}_{costante}\cdot~ n
\end{align*}
Da cui si ricava
$$
    \alpha \leq 1
$$
Quindi
$$
    \sum_{i=1}^k D^{-l_i} \leq 1
$$
\hfill $\square$



\section{Compressione Massima}
La sorgente che genera il messaggio del codice è stazionaria e senza memoria (Fig.~\vref{fig:shannon-paper}).
\begin{figure}[htb]
    \centering
    \includegraphics[width=.7\textwidth]{figures/shannon-paper.png}
    \caption{Diagramma schematico di un sistema di comunicazione, da C.~E.~Shannon, \textit{A Mathematical Theory of Communication}, Bell System Technical Journal, 1948.}
    \label{fig:shannon-paper}
\end{figure}
Poiché il messaggio codificato deve passare attraverso un canale, il quale ha una sua capacità e una sua velocità, lo si vuole comprimere il più possibile.

Ricordiamo che 
$$
    P=\{p_1,\dots,p_k\} \qquad \mathcal{A}=\{a_1,\dots,a_k\} \qquad l_i=|\varphi(a_i)| \qquad EL(\varphi) = \sum_{i=1}^k p_il_i
$$

\begin{theorem}[$\bm{1^\circ}$ Shannon]
    $$
        \varphi \text{ è UD} \quad \Rightarrow \quad EL(\varphi) \geq \mathcal{H}_D(P)
    $$

    con 
    $$
        \mathcal{H}_D(P) = \sum_{i=1}^k p_i\cdot\log_D\frac{1}{p_i}
    $$
\end{theorem}

\paragraph{Dimostrazione} 
\begin{align*}
    EL(\varphi)-\mathcal{H}_D(P) &= \sum_{i=1}^k p_i\underbrace{l_i}_{\log_DD^{l_i}} + \sum_{i=1}^k p_i\cdot\log_Dp_i\\
    &= \sum_{i=1}^k p_i\cdot\log_D(D^{l_i}\cdot p_i)\\
\end{align*}
Prima di proseguire, ricordiamo la seguente proprietà dei logaritmi su $\mathbb{N}$:
$$
    \log_ex\leq x-1; \qquad -\log_ex\geq -(x-1) 
$$
e anche la proprietà dei logaritmi:
$$
    \log_bx = \frac{\log_cx}{\log_cb}
$$
Continuiamo la dimostrazione:
\begin{align*}
    EL(\varphi)-\mathcal{H}_D(P) &= \sum_{i=1}^k p_i\cdot\log_D(D^{l_i}\cdot p_i)\\
    &= \frac{1}{log_eD} \sum_{i=1}^k p_i\cdot\log_e(D^{l_i}\cdot p_i)\\
    &= -\frac{1}{\log_eD} \sum_{i=1}^k p_i\cdot\log_e\left(\frac{1}{D^{l_i}\cdot p_i}\right)\\
    &\geq -\frac{1}{\log_eD} \sum_{i=1}^k p_i\cdot\left(\frac{1}{D^{l_i}\cdot p_i}-1\right)\\
    &= -\frac{1}{\log_eD} \underbrace{\left(\sum_{i=1}^k\frac{1}{D^{l_i}}-\underbrace{1}_{\sum p_i}\right)}_{\leq 0}\\
    &\geq 0
\end{align*}
\hfill $\square$ 


% LEZIONE 5
\section{Shannon Code}
Dal Teorema $1^\circ$ Shannon abbiamo:
$$
EL(\varphi) = \sum_{i=1}^k p_il_i ~ \geq ~ \mathcal{H}_D(P) = \sum_{i=1}^k p_i\log_D\frac{1}{p_i}
$$
La differenza sta in $l_i$ e $\log_D\frac{1}{p_i}$, quindi vogliamo provare ad eguagliarli:
$$
    l_i = \log_D\frac{1}{p_i}
$$
Ma $\log_D\frac{1}{p_i}$ non è necessariamente intero. Decidiamo quindi di considerare il suo primo intero più grande:
$$
l_i = \left\lceil\log_D\frac{1}{p_i}\right\rceil = \left\lceil-\log_Dpi\right\rceil
$$
È sempre possibile definire un codice UD con tali lunghezze? Possiamo utilizzare Kraft-McMillan. Vogliamo controllare se
$$
    \sum_{i=1}^k D^{-\lceil-\log_Dpi\rceil}
    \stackrel{?}{\leq}
    1
$$
Sappiamo che 
$$
    \lceil-\log_Dpi\rceil = -log_Dp_i+\beta_i \qquad\qquad 0\leq\beta_i<1
$$
Quindi possiamo scrivere
\begin{align*}
    \sum_{i=1}^k D^{-(-log_Dp_i+\beta_i)} &= \sum_{i=1}^k D^{log_Dp_i}\cdot D^{\beta_i}\\
    &= \sum_{i=1}^k p_i\cdot D^{-\beta_i}\\
    &= \sum_{i=1}^k p_i\cdot \frac{1}{D^{\beta_i}}
\end{align*}
Ricordiamo che $0\leq\beta_i<1$ e $D>1$, di conseguenza $\frac{1}{D^{\beta_i}}\leq 1$. Quindi
\begin{align*}
    \sum_{i=1}^k p_i\cdot \frac{1}{D^{\beta_i}} &\leq 1\\
    \sum_{i=1}^k D^{-\lceil-\log_Dpi\rceil} &\leq 1
\end{align*}
Esiste quindi un prefix code con lunghezze $l_i=\lceil-\log_Dpi\rceil$ definibile utilizzando una strategia greedy sull'albero $D$-ario.
\begin{center}
    \begin{tikzpicture}[grow=right]
        \tikzstyle{level 1}=[level distance=2cm,sibling distance=2cm]
        \tikzstyle{level 2}=[level distance=2cm,sibling distance=1.2cm]
        \node {$\bullet$}
        child {
            [dotted] node { }
            edge from parent
        }
        child {
            node {$\bullet$}
            edge from parent
        }
        child {
            node {$\bullet$}
            child {
                [dotted] node[right] { }
                edge from parent
            }
            child {
                [dotted] node[right] { }
                edge from parent
            }
            child {
                [dotted] node[right] {$\varphi(a_1)$}
                edge from parent
            }
            edge from parent
        };

        \draw [decorate,
            decoration = {brace,amplitude=10pt}] (-0.2,0.3) -- (3.8,3.5);
        \node[rotate=40] at (1.3,2.4) {$\lceil-\log_Dpi\rceil$};
    \end{tikzpicture}
\end{center}

\textcolor{Red}{TODO: finire lezione 5}


% LEZIONE 6
\begin{definition}[Efficienza (Efficiency of code)]
    $$
        Eff(\varphi) = \frac{\mathcal{H}_D(P)}{EL(\varphi)}
    $$
\end{definition}
\emph{Eff} è sempre $\leq 1$, e ci dice quanto siamo vicini all'entropia, che non è sempre raggiungibile.\medskip

Ci chiediamo perché Huffmann (merge delle foglie) è ottimale, mentre Shannon-Fano (splitting dalla radice) non lo è?


\section{Lempel-Ziv}
\textcolor{Red}{TODO: scrivere la storia}


\subsection{LZ77}


\part{Complessità}
% LEZIONE 9


\chapter{Introduzione}

Libro di Papadimitriou main reference.

In questa parte utilizzeremo come modello di computazione le \textbf{macchine di Turing} (MdT). Esistono diversi modelli di MdT: macchine di Turing multinastro, macchine di Turing input/output, macchine di Turing con oracolo, macchine di Turing nondeterministiche. Le MdT verranno utilizzate per confrontare i diversi risultati di complessità che possiamo ottenere.

Ci concentreremo sia su \textbf{complessità temporale} (time complexity) che \textbf{spaziale} (space complexity). Il focus non sarà sulla complessità di un dato algoritmo, ma sulla complessità di un problema. I \textbf{problemi} possono essere classificati come di decisione (decision problems), di funzione (function problems), o di ottimizzazione (optimization problems).
\begin{itemize}
    \item \textbf{Decision problem} $P:\text{inputs}\to\{\text{yes},\text{no}\}$
    \item \textbf{Function problem} computare una data funzione, ad esempio l'ordinamento di una lista
    \item \textbf{Optimization problem} tra tutti i possibili output, si vuole trovare quello che minimizza o massimizza una funzione di costo. 
\end{itemize}

\paragraph{Esempio} Sia $G=(V,E)$ un grafo, e $u,v\in V$ due nodi. 
\begin{itemize}
    \item[--] decidere se esiste un cammino da $u$ a $v$ è un problema di decisione
    \item[--] trovare un cammino da $u$ a $v$ è un problema di funzione
    \item[--] trovare il cammino più corto da $u$ a $v$ è un problema di ottimizzazione
\end{itemize}

In questo corso ci concentreremo sui problemi di decisione. Se si ha una soluzione per un problema di funzione o di ottimizzazione, si possiede automaticamente una soluzione per il problema di decisione.\medskip

% (disegno nelle note)
Immaginiamo tutti gli input possibili al problema dell'esempio precedente come ad un insieme infinito di tuple $(G,u,v)$. Questo insieme si può dividere in due: il sottoinsieme dei yes di tutte le codifiche binarie di triple $(G,u,v)$ tali che esiste un cammino in $G$ da $u$ a $v$, e, inversamente, il sottoinsieme no.
\begin{center}
    \begin{tikzpicture}
        \node[ellipse,
            draw,
            minimum width=6cm,
            minimum height=3cm] (A) at (0,0) { };
        \draw (0,-1.5) -- (0,1.5);
        \node[above left] at (A.north west) {$L$};
        \node[below] at (A.220) {yes};
        \node[below] at (A.-40) {no};

        \node at (-2,0.4) {$\bullet$};
        \node at (-0.8,0.2) {$\bullet$};
        \node at (-1.2,-0.4) {$\bullet$};
        \node at (-0.8,-0.8) {$\dots$};
        \node at (1.6,-0.2) {$\bullet$};
        \node at (1.2,0.4) {$\bullet$};
        \node at (0.8,-0.8) {$\dots$};

        \draw[->,decorate,decoration=snake] (-2,0.4) -- (-3.6,1);
        \node[left] at (-3.6,1) {$(G,u,v)$};
    \end{tikzpicture}
\end{center}
La codifica binaria di una tripla è una stringa del tipo $1011\dots$. Più precisamente, è una stringa sull'alfabeto $\Sigma=\{0,1\}$. L'insieme di tutte le possibili stringhe binarie è $\Sigma^*$. Questo insieme è quindi il linguaggio $L$ sottoinsieme di $\Sigma^*$, ovvero $L\subseteq\Sigma^*$.
$$
    L = \{ \bin(G,u,v)|ln~G~u\to v \}
$$

\paragraph{Esempio} Consideriamo interi rappresentati in binario. Vogliamo decidere se un dato intero $x$ è divisibile per 4.
$$
    \bin(x) = 10\dots11
$$
in questo caso non è divisibile per 4. Un numero binario è divisibile per 4 se e solo se i due bit meno significativi sono 0.
$$
    \bin(x) = x_n,x_{n-1},\dots,x_1,x_0 \quad \Leftrightarrow \quad x_0 = 0 \text{ and } x_1 = 0
$$
Il linguaggio indicato da questo problema di decisione è
$$
    L = \{ x\in\{0,1\}^* | x=x_n,x_{n-1},\dots,x_1,x_0 \land x_0 = 0 \land x_1 = 0 \}
$$


\paragraph{Esempio: palindromo} Decidere se una stringa è palindroma, con $\Sigma = \{0,1\}$.
$$
    x_1,x_2,x_3,\dots,x_3,x_2,x_1
$$
Ad esempio, $x=101$ è palindroma, mentre $x=1010$ non lo è. Cerchiamo il linguaggio $L=\{x|x \text{ è palindroma}\}$. Utilizziamo una macchina di Turing.
\begin{center}
    \begin{tikzpicture}
        % Draw the horizontal strip
        \draw (0,0) -- (5,0);
        \draw (0,.5) -- (5,.5);
        %  Draw the squares
        \foreach \x in {0.5,1,...,4.5} {
          \draw (\x,0) -- (\x,.5);
        }

        % Label the squares
        % \foreach \x in {2.25,2.75,3.25,4.25,4.75,5.25} {
        %   \node[above] at (\x,0) {$*$};
        % }
        \node[above] at (0.75,0) {$\rhd$};
        \node[above] at (1.25,0) {$x_1$};
        \node[above] at (1.75,0) {$x_2$};
        \node[above] at (3.75,0) {$x_n$};
        \node[above] at (2.8,0.1) {\Huge \dots};
        \node[above] at (4.25,0) {$\sqcup$};
        \draw [decorate,
            decoration = {brace,mirror,amplitude=5pt}] (1,-0.1) -- (4,-0.1);
        \node[below] at (2.5,-0.2) {$x$};
        \node[below] at (0.75,0) {\LARGE$\substack{\uparrow\\s}$};
        % \node at (0.75,-0.65) {$s$};
    \end{tikzpicture}
\end{center}
Si parte dallo stato $s$ e si vuole finire nello stato $p$ solo quando $x$ è palindroma. Per decidere se $x$ è palindroma, si può leggere $x_1$, ricordarne il valore nello stato del puntatore, e poi confrontarlo con $x_n$. Se sono uguali, si ripete lo stesso procedimento con $x_2$ e $x_{n-1}$, e così via. Se si arriva a $x_n$ e $x_1$ senza aver trovato una discrepanza, allora $x$ è palindroma. Se invece si trova una discrepanza, allora $x$ non è palindroma. Le transizioni sono le seguenti:
\begin{align*}
    \delta(s,\rhd) &= (q,\rhd,\to)\\
    \delta(q,1) &= (q_1,\rhd,\to)\\
    \delta(q,0) &= (q_0,\rhd,\to)\\
\end{align*}
\textcolor{Red}{TODO: finire di scrivere le transizioni}
Questa macchina eseguirà un numero quadratico di passi per controllare se la stringa $x$ è palindroma: $O(|x|^2)$.\medskip

Se si vuole controllare in C (o in un altro linguaggio) se una stringa è palindroma, si può scrivere un programma che confronta il primo e l'ultimo carattere, poi il secondo e il penultimo, e così via, eseguendo un numero lineare di passi. La complessità è $O(|x|)$. Questo è un esempio di come la complessità di un problema dipenda dal modello di computazione utilizzato.


\section{Tesi di Church-Turing Estesa} La tesi di Church-Turing afferma che ogni cosa che può essere computata, può essere computata da una macchina di Turing.

La versione estesa afferma che tutti i modelli (ragionevoli) di calcolo sono correlati polinomialmente. Questo significa che se un problema è risolvibile in tempo polinomiale in un modello di computazione, allora è risolvibile in tempo polinomiale in ogni modello di computazione.

In altre parole, la tesi di Church-Turing estesa afferma che la complessità computazionale di un problema è indipendente dal modello di calcolo utilizzato per risolverlo.
$$
    \underset{\text{problema}}{P} \to \text{MdT }O(f(n)) \to \underset{\text{modello}}{M}~O(p(f(n)))  
$$
Ma è vera anche la direzione contraria. \textcolor{Red}{TODO: ???}




%%%%%%%%%%%%%%%%%%%%%%%%%%%%
\chapter{Macchine di Turing}


\section{Definizioni}

\begin{definition}[Configurazione]
    Una configurazione è una tripla $(q,w,u)$, con
    \begin{itemize}
        \item $q\in K\cup\{\text{yes, no, halt}\}$
        \item $w,u\in\Sigma^*$
    \end{itemize}
\end{definition}
Ad esempio, graficamente, una configurazione è
\begin{center}
    \begin{tikzpicture}
        % Draw the horizontal strip
        \draw (0,0) -- (6,0);
        \draw (0,.5) -- (6,.5);
        %  Draw the squares
        \foreach \x in {0.5,1,1.5,2,3,3.5,4.5,5,5.5} {
          \draw (\x,0) -- (\x,.5);
        }

        \node[above] at (0.75,0) {$\rhd$};
        \node[above] at (1.25,0) {$x_1$};
        \node[above] at (1.75,0) {$x_2$};
        \node[above] at (2.5,0) {\dots};
        \node[above] at (3.25,0) {$x_i$};
        \node[above] at (4,0) {\dots};
        \node[above] at (4.75,0) {$x_n$};
        \node[above] at (5.25,0) {$\sqcup$};
        \draw [decorate,
            decoration = {brace,amplitude=5pt}] (0.5,0.6) -- (3.48,0.6);
        \node[above] at (2,0.7) {$w$};
        \draw [decorate,
            decoration = {brace,amplitude=5pt}] (3.52,0.6) -- (5,0.6);
        \node[above] at (4.25,0.7) {$u$};
        \node[below] at (3.25,0) {\LARGE$\substack{\uparrow\\q}$};
    \end{tikzpicture}
\end{center}


\begin{definition}[Configurazione Iniziale]
    La configurazione iniziale su una stringa $x$ è una tripla 
    $$
    (1,\rhd,x)
    $$
\end{definition}

\begin{definition}[Configurazioni Finali]
    Le configurazioni finali su una stringa $x$ sono una tripla 
    $$
    (H,w,u)
    $$
    dove $H\in\{\text{yes, no, halt}\}$.
\end{definition}


\begin{definition}[Passo di Computazione]
    $$
        (q,w,u) \overset{\delta}{\to} (q',w',u') 
    $$
\end{definition}
Ad esempio, il passo di computazione è $(s,\rhd,001)\to(q,\rhd 0,01)$
\begin{center}
    \begin{tikzpicture}
        % Draw the horizontal strip
        \draw (0,0) -- (4,0);
        \draw (0,.5) -- (4,.5);
        %  Draw the squares
        \foreach \x in {0.5,1,...,3.5} {
          \draw (\x,0) -- (\x,.5);
        };
        \node[above] at (0.75,0) {$\rhd$};
        \node[above] at (1.25,0) {$0$};
        \node[above] at (1.75,0) {$0$};
        \node[above] at (2.25,0) {$1$};
        \node[above] at (2.75,0) {$\sqcup$};
        \node[above] at (3.25,0) {$\sqcup$};
        \node[below] at (0.75,0) {\LARGE\cancel{$\substack{\uparrow\\s}$}};
        \node[below] at (1.25,0) {\LARGE$\substack{\uparrow\\q}$};
    \end{tikzpicture}
\end{center}
Eseguito applicando $\delta(s,\rhd)=(q,\rhd,\to)$.


\begin{definition}[Time Complexity per una MdT $\bm{\mathcal{M}}$ sull'input $\bm{x}$]
    $\mathcal{M}$ ha time complexity $t$ su $x$ se dopo esattamente $t$ passi si raggiunge una configurazione finale.
    $$
        (s,\rhd,x)\underbrace{\to\dots\to}_{t\text{ passi}}(H,w,u)
    $$
    Indicata in breve con $(s,\rhd,x)\to^t(H,w,u)$.\medskip

    \noindent $\mathcal{M}$ ha time complexity $f:\mathbb{N}\to\mathbb{N}$ se, $\forall x\in\Sigma^*$, $(s,\rhd,x)\to^t(H,w,u)$ con $t\leq f(|x|)$.
\end{definition}
La dimensione dell'input (bit length dell'input) è $|x|$. Questa è una complessità nel caso peggiore ($\leq$). Non stiamo utilizzando la notazione big-O.



\section{Unlimited Register Machines}
Una Unlimited Register Machine (URM) è una macchina di Turing con un numero illimitato di registri. 
\begin{table}[H]
    \centering
    \begin{tabular}{c|c|}
    \cline{2-2}
    $R_0$ & $r_0$ \\ \cline{2-2} 
    $R_1$ & $r_1$ \\ \cline{2-2} 
          & \dots \\ \cline{2-2} 
    $R_m$ & $r_m$ \\ \cline{2-2} 
          & \dots 
    \end{tabular}
\end{table}
Ogni registro contiene un numero naturale. Quindi, il contenuto del registro $R_m$ sarà $r_m\in\mathbb{N}$. Le operazioni possibili sono:
\begin{itemize}
    \item \textbf{incremento} $S(i)$: $r_i:=r_i+1$
    \item \textbf{azzeramento} $Z(i)$: $r_i:=0$
    \item \textbf{trasferimento} $T(i,j)$: $r_j:=r_i$, ovvero trasferisco il contenuto del registro $R_i$ nel registro $R_j$
    \item \textbf{jump} $J(i,j,k)$: se $r_i=r_j$ allora salta all'istruzione $k$, altrimenti prosegue con l'istruzione successiva
\end{itemize}

% Macchina con un unbounded number of registers.

% All'inizio, l'input $x$ è in $R_0$. Abbiamo una specie di counter.

\paragraph{Esempio} Dati $x,y\in\mathbb{N}$, decidere se $x=y$.

\subparagraph{MdT} Si può utilizzare una macchina di Turing che contiene la rappresentazione binaria dei due interi, separati da un separatore.
\begin{center}
    \begin{tikzpicture}
        % Draw the horizontal strip
        \draw (0,0) -- (6.5,0);
        \draw (0,.5) -- (6.5,.5);
        %  Draw the squares
        \foreach \x in {0.5,1,1.5,2.5,3,3.5,4,5,5.5,6} {
          \draw (\x,0) -- (\x,.5);
        };
        \node[above] at (0.75,0) {$\rhd$};
        \node[above] at (1.25,0) {$x_1$};
        \node[above] at (2,0) {\dots};
        \node[above] at (2.75,0) {$x_n$};
        \node[above] at (3.25,0) {;};
        \node[above] at (3.75,0) {$y_1$};
        \node[above] at (4.5,0) {\dots};
        \node[above] at (5.25,0) {$y_m$};
        \node[above] at (5.75,0) {$\sqcup$};
        \node[below] at (0.75,0) {\LARGE$\substack{\uparrow\\s}$};
        \draw [decorate,
            decoration = {brace,amplitude=5pt}] (1,0.6) -- (3,0.6);
        \node[above] at (2,0.7) {$x$};
        \draw [decorate,
            decoration = {brace,amplitude=5pt}] (3.5,0.6) -- (5.5,0.6);
        \node[above] at (4.5,0.7) {$y$};
    \end{tikzpicture}
\end{center}
Questa macchina richiede, nel caso peggiore, un numero quadratico di passi per terminare. La com\-ples\-si\-tà è $\Theta(|x|^2)$.

\subparagraph{URM} Possiamo utilizzare una URM con $x$ e $y$ rispettivamente nei registri $R_0$ e $R_1$.
\begin{table}[H]
    \centering
    \begin{tabular}{c|c|}
    \cline{2-2}
    $R_0$ & $x$ \\ \cline{2-2} 
    $R_1$ & $y$ \\ \cline{2-2} 
          &    
    \end{tabular}
\end{table}
\noindent Alla fine, scriveremo 1 in $R_0$ se $x=y$, 0 altrimenti. Le istruzioni sono le seguenti:
\begin{enumerate}
    \item $J(0,1,4)$
    \item $Z(0)$
    \item $J(0,0,100)$
    \item $Z(0)$
    \item $S(0)$
\end{enumerate}
In questo caso, la complessità si può calcolare in due modi.

\begin{definition}[Time Complexity su URM]~
    \begin{itemize}
        \item \emph{Uniform cost criterium} (criterio del costo uniforme): numero di istruzioni eseguite.
        \item \emph{Logarithmic cost criterium} (criterio del costo logaritmico): ogni istruzione ha un costo proporzionale al numero di cifre coinvolte.
    \end{itemize}
\end{definition}
Quindi, per questa macchina, la complessità è 
\begin{itemize}
    \item utilizzando il criterio del costo uniforme: $\Theta(1)$
    \item utilizzando il criterio del costo logaritmico: $\Theta(|x|+|y|)$
\end{itemize}
Nel secondo caso, ci si avvicina al costo per la macchina di Turing. 

Mentre le macchine di Turing sono un modello di computazione sequenziale, nelle URM si ha l'istruzione \emph{jump}. % LEZIONE 10
In altre parole:
\begin{itemize}
    \item \textbf{MdT} 1 bit di informazione in ogni cella $\to$ tempo: numero di passi
    \item \textbf{URM} registri, un intero di lunghezza arbitraria (più bit) in ogni registro $\to$ tempo: numero di istruzioni (uniform time complexity)
\end{itemize}

\paragraph{Esempio} Computare $x+1$, $x\in\mathbb{N}$.
\subparagraph{MdT} Si ha una macchina di Turing che contiene $x$ in binario.
\begin{center}
    \begin{tikzpicture}
        % Draw the horizontal strip
        \draw (0,0) -- (4,0);
        \draw (0,.5) -- (4,.5);
        %  Draw the squares
        \foreach \x in {0.5,1,1.5,2.5,3,3.5} {
          \draw (\x,0) -- (\x,.5);
        };
        \node[above] at (0.75,0) {$\rhd$};
        \node[above] at (1.25,0) {$x_1$};
        \node[above] at (2,0) {\dots};
        \node[above] at (2.75,0) {$x_n$};
        \node[above] at (3.25,0) {$\sqcup$};
        % \node[below] at (0.75,0) {\LARGE$\substack{\uparrow\\s}$};
        \draw [decorate,
            decoration = {brace,amplitude=5pt}] (1,0.6) -- (3,0.6);
        \node[above] at (2,0.7) {$x$};
    \end{tikzpicture}
\end{center}
Nel caso peggiore $x=111\dots1$, quindi la complessità è lineare $\Theta(n)$.

\subparagraph{URM} Si ha una URM con $x$ nel registro $R_0$. È suffieciente una singola istruzione $S(0)$, quindi la complessità è $\Theta(1)$.


\subsection{URM + Prodotto}
Cambiamo il modello di computazione URM, considerando URM + prodotto. Oltre alle istruzioni $S(i)$, $Z(i)$, $T(i,j)$, e $J(i,j,k)$, aggiungiamo l'istruzione $P(i)$, che esegue l'operazione $r_i:=r_i* r_i$.

\paragraph{Esempio di programma per URM + prodotto} Abbiamo in input un numero $x$, che copiamo anche in $R_1$. Applichiamo il prodotto sul contenuto del registro $R_0$ per $x$ volte. In altre parole, vogliamo calcolare $x^{2^x}$.
\begin{enumerate}
    \item $J(1,2,5)$
    \item $P(0)$
    \item $S(2)$
    \item $J(3,3,1)$
\end{enumerate}
Pertendo da un input di $x$ in $R_0$, $x$ in $R_1$, e $0$ in tutti gli altri registri. In un generico passo di iterazione $i$ si avrà:
\begin{table}[H]
    \centering
    \def\arraystretch{1.3}
    \begin{tabular}{c|c|cc|c|cc|c|cc|c|cccc|c|c}
    \cline{2-2} \cline{5-5} \cline{8-8} \cline{11-11} \cline{16-16}
    $R_0$ & $x$      &       & $R_0$ & $x^2$    &       & $R_0$ & $(x^2)^2$ &       & $R_0$ & $(x^4)^2$ &       &         &       & $R_0$ & $x^{2^i}$  \\ \cline{2-2} \cline{5-5} \cline{8-8} \cline{11-11} \cline{16-16} 
    $R_1$ & $x$      &       & $R_1$ & $x$      &       & $R_1$ & $x$       &       & $R_1$ & $x$       &       &         &       & $R_1$ & $x$      \\ \cline{2-2} \cline{5-5} \cline{8-8} \cline{11-11} \cline{16-16} 
    $R_2$ & 0        & ~ $\to$ ~ & $R_2$ & 1        & ~ $\to$ ~ & $R_2$ & 2         & ~ $\to$ ~ & $R_2$ & 3         & ~ $\to$ & $\dots$ & $\to$ ~ & $R_2$ & $i$    \\ \cline{2-2} \cline{5-5} \cline{8-8} \cline{11-11} \cline{16-16} 
    $R_3$ & 0        &       & $R_3$ & 0        &       & $R_3$ & 0         &       & $R_3$ & 0         &       &         &       & $R_3$ & 0        \\ \cline{2-2} \cline{5-5} \cline{8-8} \cline{11-11} \cline{16-16} 
    $R_4$ & $\vdots$ &       & $R_4$ & $\vdots$ &       & $R_4$ & $\vdots$  &       & $R_4$ & $\vdots$  &       &         &       & $R_4$ & $\vdots$
    \end{tabular}
\end{table}
Il numero di istruzioni è lineare $\Theta(n)$.

\subparagraph{MdT} Se si eseguisse la stessa computazione su una macchina di Turing, si avrebbe 
\begin{center}
    \begin{tikzpicture}
        % Draw the horizontal strip
        \draw (0,2) -- (4,2);
        \draw (0,2.5) -- (4,2.5);
        %  Draw the squares
        \foreach \x in {0.5,1,1.5,2.5,3,3.5} {
          \draw (\x,2) -- (\x,2.5);
        };
        \node[above] at (0.75,2) {$\rhd$};
        \node[above] at (1.25,2) {$x_1$};
        \node[above] at (2,2) {\dots};
        \node[above] at (2.75,2) {$x_n$};
        \node[above] at (3.25,2) {$\sqcup$};
        \draw [decorate,
            decoration = {brace,amplitude=5pt}] (1,2.6) -- (3,2.6);
        \node[above] at (2,2.7) {$x$};

        \node at (2,1.3) {$\vdots$};

        % Draw the horizontal strip
        \draw (0,0) -- (4,0);
        \draw (0,.5) -- (4,.5);
        %  Draw the squares
        \foreach \x in {0.5,1,1.5,2.5,3,3.5} {
          \draw (\x,0) -- (\x,.5);
        };
        \node[above] at (0.75,0) {$\rhd$};
        \node[above] at (1.25,0) {$y_1$};
        \node[above] at (2,0) {\dots};
        \node[above] at (2.75,0) {$y_m$};
        \node[above] at (3.25,0) {$\sqcup$};
        \draw [decorate,
            decoration = {brace,mirror,amplitude=5pt}] (1,-.1) -- (3,-.1);
        \node[below] at (2,-.2) {$x^{2^x}$};
    \end{tikzpicture}
\end{center}
Quindi $\Omega(\log(x^{2^x}))=\Omega(2^x \log(x))$.\bigskip

Questo risultato sembra contraddire la tesi di Church-Turing estesa, che afferma che tutti i modelli \textbf{ragionevoli} di computazione sono correlati polinomialmente. Ma cosa significa \emph{ragionevole}? Non si può avere una operazione che fa crescere ``troppo'' l'input (nell'esempio, il prodotto), si deve utilizzare il criterio logaritmico.

In altre parole, se l'algoritmo utilizza operazioni che in un numero polinomiale di passi fanno crescere l'input esponenzialmente, e queste sono utilizzate un numero di volte che dipende dalla dimensione dell'input, allora si deve utilizzare un criterio logaritmico. Quando non si è sicuri della potenza delle operazioni della macchina, il costo di ogni singola operazione dev'essere proporzionale al numero di bit manipolati.
\begin{table}[H]
    \centering
    \def\arraystretch{1.3}
    \begin{tabular}{ccc}
    \rowcolor[HTML]{C0C0C0} 
    istruzione & uniform     & logarithmic                        \\
    $S(i)$     & $\Theta(1)$ & $\Theta(\log(r_i))$                \\
    \rowcolor[HTML]{EFEFEF} 
    $Z(i)$     & $\Theta(1)$ & $\Theta(1)$                        \\
    $T(i,j)$   & $\Theta(1)$ & $\Theta(\log(r_i))$                \\
    \rowcolor[HTML]{EFEFEF} 
    $J(i,j,k)$ & $\Theta(1)$ & $\Theta(\min(\log(r_i),\log(r_j)))$\\
    $P(i)$     & $\Theta(1)$ & $\Theta((\log(r_i))^2)$            
    \end{tabular}
\end{table}
Con $r_i$ contenuto del registro $i$. In particolare per $P(i)$, nella moltiplicazione di un numero $x$ per se stesso si ha $x_1,x_2,\dots,x_n \times x_1,x_2,\dots,x_n$. Si hanno $x^n$ bit operazioni, quindi $O((\log(x))^2)$.



\section{Ulteriori Definizioni}
Come abbiamo visto, nei problemi di decisione si ha un input $x\in\Sigma^*$ e un output in $\{\text{yes},\text{no}\}$. Possiamo definire un linguaggio $L$ come l'insieme di tutte le stringhe che hanno output yes. 
$$
    L \subseteq (\Sigma \backslash \{ \sqcup \} )^*
$$
Un problema $P$ è una funzione
$$
    P:\Sigma^*\to\{\text{yes},\text{no}\}
$$

\begin{definition}[Linguaggio Ricorsivo]
    \begin{eqnarray*}
        &\text{Una macchina di Turing $\mathcal{M}$ decide un linguaggio $L$}&\\
        &\Updownarrow&\\
        &\forall x\in(\Sigma\backslash\{\sqcup\})^* \begin{cases*}
            x\in L \to \mathcal{M}(x)=\text{yes}\\
            x\notin L \to \mathcal{M}(x)=\text{no}
        \end{cases*}&
    \end{eqnarray*}
    Il linguaggio $L$ si dice \textbf{ricorsivo}.
\end{definition}

\begin{definition}[Linguaggio Ricorsivamente Enumerabile]
    \begin{eqnarray*}
        &\text{Una macchina di Turing $\mathcal{M}$ accetta un linguaggio $L$}&\\
        &\Updownarrow&\\
        &\forall x\in(\Sigma\backslash\{\sqcup\})^* \begin{cases*}
            x\in L \to \mathcal{M}(x)=\text{yes}\\
            x\notin L \to \mathcal{M}(x)\uparrow \text{ (non termina)}
        \end{cases*}&
    \end{eqnarray*}
    Il linguaggio $L$ si dice \textbf{ricorsivamente enumerabile}.
\end{definition}

\begin{theorem}
    $$
        \text{$L$ è ricorsivo } \Rightarrow \text{ $L$ è ricorsivamente enumerabile}
    $$
\end{theorem}

\paragraph{Esempio} Trovare un linguaggio $L$ tale che $L$ è ricorsivamente enumerabile ma non ricorsivo.

Nell'halting problem abbiamo 
$$
    \mathcal{U}(\mathcal{M};x)=\mathcal{M}(x)
$$
L'halting language
$$
    H = \{ (\bin(\mathcal{M});x) ~|~ \mathcal{M}(x)\downarrow \}
$$
è ricorsivamente enumerabile ma non ricorsivo. Infatti, se $\mathcal{M}$ termina su $x$, allora $\mathcal{U}(\mathcal{M};x)=\mathcal{M}(x)=\text{yes}$, altrimenti $\mathcal{U}(\mathcal{M};x)\uparrow$. Questo è un risultato qualitativo.

\paragraph{Esempio} Sia
$$
    L = \{ \bin(\mathcal{M}) ~|~ \forall x~\mathcal{M}(x)\downarrow \text{ in al massimo 100 passi} \}
$$
$L$ è ricorsivo. Infatti, la macchina $\mathcal{M}$ può eseguire al massimo 100 spostamenti a destra sul nastro. Quindi, tutte le macchine che terminano in al massimo 100 passi accettano input $\forall x\in|\Sigma|^n$ con $n\leq 100$.

\begin{definition}[Computazione di Funzioni]
    Sia $f$ una funzione $f:(\Sigma\backslash\sqcup)^*\to\Sigma^*$. Una macchina di Turing $\mathcal{M}$ computa $f$ se
    $$
        \forall x\in(\Sigma\backslash\sqcup)^*\qquad \mathcal{M}(x)\downarrow \text{ e alla fine $f(x)$ è sul nastro}
    $$
    La funzione $f$ è detta \textbf{ricorsiva}, o \textbf{computabile}.
\end{definition}


\section{Macchine di Turing a $k$-nastri e Input/Output}
\begin{definition}[Macchina di Turing a $k$-nastri]
    Una macchina di Turing a $k$-nastri è u\-na tupla $\mathcal{M}=(K,\Sigma,\delta,s)$ con $K,\Sigma,s$ definite come per una macchina di Turing, e
    $$
        \delta : K\times\Sigma \to (K\cup\{\text{yes},\text{no},\text{halt}\}) \times
                                    (\Sigma \times \{\gets,\to,-\})^k
    $$    
\end{definition}
Una macchina di Turing a $k$-nastri è una macchina di Turing con un numero limitato di nastri, che possono essere utilizzati in parallelo. La funzione $\delta$ cambia perché si ha un puntatore per nastro.
\begin{center}
    \begin{tikzpicture}
        \draw (0,2) -- (4,2);
        \draw (0,2.5) -- (4,2.5);
        \foreach \x in {0.5,1,1.5,2} {
          \draw (\x,2) -- (\x,2.5);
        };
        \node[above] at (0.75,2) {$\rhd$};
        \node[above] at (2.5,2) {\dots};
        \node[below] at (1.25,2) {$\uparrow$};

        \draw (0,.75) -- (4,.75);
        \draw (0,1.25) -- (4,1.25);
        \foreach \x in {0.5,1,1.5,2} {
          \draw (\x,.75) -- (\x,1.25);
        };
        \node[above] at (0.75,.75) {$\rhd$};
        \node[above] at (2.5,.75) {\dots};
        \node[below] at (3.25,.75) {$\uparrow$};

        \node at (2,.2) {$\vdots$};

        \draw (0,-.5) -- (4,-.5);
        \draw (0,-1) -- (4,-1);
        \foreach \x in {0.5,1,1.5,2} {
          \draw (\x,-.5) -- (\x,-1);
        };
        \node[above] at (0.75,-1) {$\rhd$};
        \node[above] at (2.5,-1) {\dots};
        \node[below] at (0.75,-1) {$\uparrow$};

        \draw [decorate,
            decoration = {brace,amplitude=5pt}] (-.3,-1.5) -- (-.3,2.5);
        \node at (-.7,.5) {$k$};
    \end{tikzpicture}
\end{center}

\begin{definition}[Macchina di Turing a $k$-nastri con Input/Output]
    Una macchina di\\Turing a $k$-nastri con I/O è una macchina di Turing a $k$-nastri con un nastro di input e un nastro di output. Il nastro di input è di sola lettura, il nastro di output è di sola scrittura.
\end{definition}
\begin{center}
    \begin{tikzpicture}
        \draw (0,2) -- (4,2);
        \draw (0,2.5) -- (4,2.5);
        \foreach \x in {0.5,1,1.5,2} {
          \draw (\x,2) -- (\x,2.5);
        };
        \node[above] at (0.75,2) {$\rhd$};
        \node[above] at (2.5,2) {\dots};
        \node[below] at (1.25,2) {$\uparrow$};

        \draw (0,.75) -- (4,.75);
        \draw (0,1.25) -- (4,1.25);
        \foreach \x in {0.5,1,1.5,2} {
          \draw (\x,.75) -- (\x,1.25);
        };
        \node[above] at (0.75,.75) {$\rhd$};
        \node[above] at (2.5,.75) {\dots};
        \node[below] at (3.25,.75) {$\uparrow$};

        \node at (2,.2) {$\vdots$};

        \draw (0,-.5) -- (4,-.5);
        \draw (0,-1) -- (4,-1);
        \foreach \x in {0.5,1,1.5,2} {
          \draw (\x,-.5) -- (\x,-1);
        };
        \node[above] at (0.75,-1) {$\rhd$};
        \node[above] at (2.5,-1) {\dots};
        \node[below] at (0.75,-1) {$\uparrow$};

        \draw [decorate,
            decoration = {brace,amplitude=5pt}] (-.3,-1.5) -- (-.3,2.5);
        \node at (-.7,.5) {$k$};

        \node[right] at (4.25,2.25) {input};
        \node[right] at (4.25,-.75) {output};
    \end{tikzpicture}
\end{center}

\begin{definition}[Configurazione e Configurazione Iniziale]
    Siano $w_i,u_i\in\Sigma^*$ stringhe. Una configurazione è una tupla
    $$
        (q,w_1,u_1,w_2,u_2,\dots,w_k,u_k)
        \to 
        (q',w_1',u_1',w_2',u_2',\dots,w_k',u_k')
    $$
    Una configurazione iniziale su input $x$ è una tupla
    $$
        (s,\rhd,x,\rhd,\varepsilon,\dots,\rhd,\varepsilon)
    $$
\end{definition}



% LEZIONE 11
\textcolor{Red}{TODO: lezione 11}


% LEZIONE 12
% \chapter{Complessità Spaziale}
\subsection{Complessità Spaziale}
\begin{definition}[Complessità Spaziale per una MdT a $k$-nastri su input $x$]
    Si ha che
    $$
        (s,\rhd,x) \to^* (H,w_1,u_1,\dots,w_k,u_k)
    $$
    con $H=\{\text{halt},\text{yes},\text{no}\}$. Lo spazio utilizzato è
    $$
        \sum_{i=1}^k |w_i| + |u_i|
    $$
\end{definition}

\begin{definition}[Complessità Spaziale per una MdT a $k$-nastri con I/O]
    Si ha che
    $$
        (s,\rhd,x) \to^* (H,w_1,u_1,\dots,w_k,u_k)
    $$
    con $H=\{\text{halt},\text{yes},\text{no}\}$. Lo spazio utilizzato è
    $$
        \sum_{i=1}^{k-1} |w_i| + |u_i|
    $$
\end{definition}
dove $\delta(q,\sigma_1,\dots,\sigma_k)=(q',\sigma_1',\dots,\sigma_k',\to)$.

\begin{definition}[Classi di Complessità Spaziale]
    $L$ è decidibile in spazio $f(n)$ se esiste una macchina di Turing a $k$-nastri con I/O $\mathcal{M}$ che decide $L$ e, $\forall x$, $\mathcal{M}$ utilizza uno spazio al massimo $f(|x|)$.
\end{definition}

\paragraph{Esempio: palindromo} $L=\{x|x\text{ è palindroma}\}$. Si vuole trovare la macchina più efficiente in termini di spazio. La seguente macchina è efficiente nel tempo:
\begin{center}
    \begin{tikzpicture}
        \draw (0,1) -- (4.5,1);
        \draw (0,1.5) -- (4.5,1.5);
        \foreach \x in {0.5,1,1.5,3,3.5,4} {
          \draw (\x,1) -- (\x,1.5);
        };
        \node[above] at (0.75,1) {$\rhd$};
        \node[above] at (1.25,1) {$x_1$};
        \node[above] at (2.25,1) {\dots};
        \node[above] at (3.25,1) {$x_n$};
        \node[above] at (3.75,1) {$\sqcup$};

        \draw (0,0) -- (4.5,0);
        \draw (0,.5) -- (4.5,.5);
        \foreach \x in {0.5,1,1.5,3,3.5,4} {
          \draw (\x,0) -- (\x,.5);
        };
        \node[above] at (0.75,0) {$\rhd$};
        \node[above] at (1.25,0) {$x_1$};
        \node[above] at (2.25,0) {\dots};
        \node[above] at (3.25,0) {$x_n$};
        \node[above] at (3.75,0) {$\sqcup$};

        \draw[->] (4.6,1.25) to[out=0,in=0,distance=20] (4.6,.25);
        \node[right] at (5.1,.75) {copia};

        \node[left] at (-.25,1.25) {input};
        \node[left] at (-.25,.25) {working tape};
    \end{tikzpicture}
\end{center}
perché ha $\text{TIME }\Theta(n)$ e $\text{SPACE }\Theta(n)$. Mentre la seguente macchina è efficiente nello spazio:
\begin{center}
    \begin{tikzpicture}
        \draw (0,1) -- (5.5,1);
        \draw (0,1.5) -- (5.5,1.5);
        \foreach \x in {0.5,1,1.5,4,4.5,5} {
          \draw (\x,1) -- (\x,1.5);
        };
        \node[above] at (0.75,1) {$\rhd$};
        \node[above] at (1.25,1) {$\cancel{x_1}$};
        \node[above] at (1.25,1.5) {$\rhd$};
        \node[above] at (2.75,1) {\dots};
        \node[above] at (4.25,1) {$\cancel{x_n}$};
        \node[above] at (4.25,1.5) {$\sqcup$};
        \node[above] at (4.75,1) {$\sqcup$};

        \draw (0,0) -- (5.5,0);
        \draw (0,.5) -- (5.5,.5);
        \foreach \x in {0.5,1,1.5,2.5,3,4,4.5,5} {
          \draw (\x,0) -- (\x,.5);
        };
        \node[above] at (0.75,0) {$\rhd$};
        \node[above] at (1.25,0) {$x_1$};
        \node[above] at (2,0) {\dots};
        \node[above] at (2.75,0) {$x$};
        \node[above] at (3.5,0) {\dots};
        \node[above] at (4.25,0) {$x_n$};
        \node[above] at (4.75,0) {$\sqcup$};

        \draw (0,-1) -- (5.5,-1);
        \draw (0,-.5) -- (5.5,-.5);
        \foreach \x in {0.5,1,1.5,2.5,3,4,4.5,5} {
          \draw (\x,-1) -- (\x,-.5);
        };
        \node[above] at (0.75,-1) {$\rhd$};
        \node[above] at (1.25,-1) {$x_1$};
        \node[above] at (2,-1) {\dots};
        \node[above] at (2.75,-1) {$x$};
        \node[above] at (3.5,-1) {\dots};
        \node[above] at (4.25,-1) {$x_n$};
        \node[above] at (4.75,-1) {$\sqcup$};

        \draw [decorate,
            decoration = {brace,mirror,amplitude=5pt}] (.5,-1.1) -- (3,-1.1);
        \node[below] at (1.75,-1.25) {$y$};
    \end{tikzpicture}
\end{center}
con $\text{SPACE }\Theta(\log n)$.\bigskip

\begin{definition}
    \begin{align*}
        \text{TIME}(f(n)) = \{ L~|~L\text{ può essere deciso in tempo }f(n) \}\\
        \text{SPACE}(f(n)) = \{ L~|~L\text{ può essere deciso in spazio }f(n) \}
    \end{align*}
\end{definition}
In altre parole, $\text{SPACE}(f(n))$ è l'insieme di tutti i linguaggi che possono essere decisi in tempo $f(n)$ da una macchina di Turing a $k$-nastri con I/O. Per ogni input $x$ tale che $|x|=n$, la macchina utilizza spazio al più $f(n)$. 
\begin{property}
    Se esiste una macchina di Turing che decide $L$ in tempo $f(n)$, e $f(n)\geq n$, al\-lo\-ra esiste una macchina di Turing con I/O che decide $L$ in tempo $O(f(n))$.
\end{property}

\paragraph{Esempio} Calcola $x+y$.
\begin{center}
    \begin{tikzpicture}
        \draw (0,1) -- (7,1);
        \draw (0,1.5) -- (7,1.5);
        \foreach \x in {0.5,1,1.5,2,3,3.5,4,4.5,5.5,6,6.5} {
          \draw (\x,1) -- (\x,1.5);
        };
        \node[above] at (0.75,1) {$\rhd$};
        \node[above] at (1.25,1) {$x_1$};
        \node[above] at (1.75,1) {$x_2$};
        \node[above] at (2.5,1) {\dots};
        \node[above] at (3.25,1) {$x_r$};
        \node[above] at (3.75,1) {;};
        \node[above] at (4.25,1) {$y_1$};
        \node[above] at (5,1) {\dots};
        \node[above] at (5.75,1) {$y_m$};
        \node[above] at (6.25,1) {$\sqcup$};

        \draw [decorate,
            decoration = {brace,amplitude=5pt}] (1,1.6) -- (6,1.6);
        \node[above] at (3.5,1.75) {$n$};

        \node at (3.5,0.3) {\vdots};

        \draw (0,-1) -- (7,-1);
        \draw (0,-.5) -- (7,-.5);
        \foreach \x in {0.5,1,1.5,5.5,6,6.5} {
          \draw (\x,-1) -- (\x,-.5);
        };
        \node[above] at (0.75,-1) {$\rhd$};
        \node[above] at (1.25,-1) {$z_1$};
        \node[above] at (3.5,-1) {\dots};
        \node[above] at (5.75,-1) {$z_n$};
        \node[above] at (6.25,-1) {$\sqcup$};

        \draw [decorate,
            decoration = {brace,amplitude=5pt}] (-.2,-.3) -- (-.2,.8);
        \node[left] at (-.35,.25) {working tapes};

        \node[right] at (7.2,1.25) {input};
        \node[right] at (7.2,-.75) {output};
    \end{tikzpicture}
\end{center}
Questo ha spazio lineare $\Theta(n)$ (molto male).\bigskip

\begin{definition}[Classe P]
    Definiamo la classe P come
    $$
        \text{P} = \bigcup_{h\in\mathbb{N}} \text{TIME}(n^h)
    $$
    ovvero l'unione di tutti i problemi che possono essere risolti in tempo polinomiale.
\end{definition}
La classe P ci piace così tanto perché abbiamo la tesi di Church-Turing estesa. Questa classe è \textbf{invariante} rispetto alla scelta del modello di computazione.
Possiamo definire la classe EXP
$$
    \text{EXP} = \bigcup_{h\in\mathbb{N}} \text{TIME}(2^{n^h})
$$

La classe $\mathbb{L}$, PSPACE, e EXPSPACE
\begin{align*}
    \mathbb{L} &= \text{SPACE}(\log n)\\
    \text{PSPACE} &= \bigcup_{h\in\mathbb{N}} \text{SPACE}(n^h)\\
    \text{EXPSPACE} &= \bigcup_{h\in\mathbb{N}} \text{SPACE}(2^{n^h}) 
\end{align*}

\begin{property}~
    $$
        \text{TIME}(f(n)) \subseteq \text{SPACE}(f(n))
    $$    
\end{property}

% book, section 2.6
\section{Random Access Machines}
Capitolo 2.6 del libro. Le random access machine (RAM), sono un modello di computazione sequenziale, composte da registri di input e registri di lavoro. Ogni registro contiene un intero.
\begin{table}[H]
    \centering
    \def\arraystretch{1.5}
    \begin{tabular}{lc|c|clc|c|}
        \cline{3-3} \cline{7-7}
        registri di input & $I_1$    & $\qquad$ & $\qquad$ & working registers & $R_0$    & $\qquad$ \\ \cline{3-3} \cline{7-7} 
                          & $\vdots$ &          &          &                   & $\vdots$ &          \\ \cline{3-3} \cline{7-7} 
                          & $R_i$    &          &          &                   & $I_j$    &          \\ \cline{3-3} \cline{7-7} 
                          & $\vdots$ &          &          &                   & $\vdots$ &         
    \end{tabular}
\end{table}

\noindent Le operazioni possibili sono:
\begin{itemize}
    \item \texttt{READ} $j$: $r_0:=i_j$
    \item \texttt{READ} $\uparrow j$: $r_0:=i_{r_j}$ (vai al registro $R_j$, leggine il contenuto $h$, vai al registro $I_h$, copiane il contenuto in $R_0$)
    \item \texttt{STORE} $j$: $r_j:=r_0$
    \item \texttt{STORE} $\uparrow j$
    \item \texttt{LOAD} $j$: $r_0:=r_j$
    \item \texttt{LOAD} $\uparrow j$
    \item \texttt{LOAD} $=j$: $r_0:=j$
    \item \texttt{ADD} $j$: $r_0:=r_0+r_j$
    \item \texttt{ADD} $\uparrow j$: $r_0:=r_0+r_{r_j}$ 
    \item \texttt{ADD} $=j$
    \item \texttt{SUB} $j$
    \item \dots
    \item \texttt{HALF}: $r_0:=\left\lfloor \dfrac{r_0}{2} \right\rfloor$ (tolgo da $r_0$ l'ultimo bit)
    \item \texttt{JUMP} $j$: $k:=j$ (contatore)
    \item \texttt{JPOS} $j$: if $r_0>0$ then $k:=j$
    \item \texttt{JNEG} $j$
    \item \texttt{JZERO} $j$
    \item \texttt{HALT}
\end{itemize}
Il libro dimostra che
\begin{theorem}
    RAM con complessità temporale uniforme e macchine di Turing con $k$-nastri sono correlate polinomialmente.
\end{theorem}
In particolare 
\begin{align*}
    \underbrace{\text{MdT}}_{f(n)} &\to^{\text{simula}} \underbrace{\text{RAM}}_{O(f(n))}\\
    \underbrace{\text{RAM}}_{f(n)} &\to^{\text{simula}} \underbrace{\text{MdT a 7-nastri}}_{O((f(n))^3)}\\
\end{align*}
Ad esempio, quando si sommano due numeri, si ottiene al massimo 1 bit in più dell'input maggiore.

% LEZIONE 13
\section{Macchine Nondeterministiche}
Si hanno
\begin{itemize}
    \item Macchine deterministiche $\mathcal{M}=(K,\Sigma,\delta,s)$, con $\delta$ \textbf{funzione}
    $$
        \delta:K\times\Sigma^k\to (K\cup\{\text{yes},\text{no},\text{halt}\}) \times \Sigma^k \times \{\gets,\to,-\}^k
    $$
    la cui configurazione è del tipo
    $$
        c \to c'
    $$
    \item Macchine nondeterministiche $\mathcal{N}=(K,\Sigma,\Delta,s)$, con $\Delta$ \textbf{relazione}
    $$
        \Delta\subseteq K\times\Sigma^k\times (K\cup\{\text{yes},\text{no},\text{halt}\}) \times \Sigma^k \times \{\gets,\to,-\}^k
    $$
    quindi con una o più possibili transizioni. La configurazione $(q,u_1,w_1,\dots,q_k,w_k)$ è del tipo 
    \begin{center}
        \begin{tikzpicture}[grow=right, ->, level distance=1cm,
            level 1/.style={sibling distance=1cm}]
            \node {$c$}
              child {node {$c'''$}}
              child {node {$c''$}}
              child {node {$c'$}};
        \end{tikzpicture}
    \end{center}    
\end{itemize}

\paragraph{Esempio: Reachability Problem} Dato un grafo diretto $G=(V,E)$, e due nodi $u,v\in V$, decidere se $u$ raggiunge $v$ (se esiste un cammino da $u$ a $v$). Studiamo la complessità in tempo e spazio di questo problema utilizzando sia un modello deterministico che nondeterministico.
\subparagraph{Modello deterministico} Un possibile algoritmo per risolvere questo problema è BFS$(G,u)$. Si costruisce un albero con radice $u$, e ad ogni livello si aggiungono i nodi raggiungibili in un passo. Utilizzando dei colori, alla fine della visita tutti i nodi visitati saranno neri, e quelli non raggiungibili grigi: è sufficiente controllare se $v$ è nero. Se $v$ è raggiungibile da $u$, allora $v$ sarà raggiunto da $u$ in un numero di passi $\leq |V|$. Quindi, la \textbf{complessità in tempo} è $O(|V|+|E|)$, ovvero lineare rispetto alla dimensione del grafo. Con una macchina di Turing:
\begin{center}
    \begin{tikzpicture}
        \draw (0,1) -- (7,1);
        \draw (0,1.5) -- (7,1.5);
        \foreach \x in {0.5,1} {
          \draw (\x,1) -- (\x,1.5);
        };
        \node[above] at (0.75,1) {$\rhd$};
        \node[above] at (3.75,1) {$\dots$};
        \draw [decorate,
            decoration = {brace,amplitude=5pt}] (1,1.6) -- (3.95,1.6);
        \node[above] at (2.5,1.75) {$G$};
        \draw [decorate,
            decoration = {brace,amplitude=5pt}] (4.05,1.6) -- (4.95,1.6);
        \node[above] at (4.5,1.75) {$u$};
        \draw [decorate,
            decoration = {brace,amplitude=5pt}] (5.05,1.6) -- (6,1.6);
        \node[above] at (5.5,1.75) {$v$};
        \draw [decorate,
            decoration = {brace,mirror,amplitude=5pt}] (1,.9) -- (6,.9);
        \node[below] at (3.5,.75) {$n$};

        \draw (0,-1) -- (7,-1);
        \draw (0,-.5) -- (7,-.5);
        \foreach \x in {0.5,1} {
          \draw (\x,-1) -- (\x,-.5);
        };
        \node[above] at (0.75,-1) {$\rhd$};
        \node[above] at (3.75,-1) {$\dots$};
        \draw [decorate,
            decoration = {brace,amplitude=5pt}] (1,-.4) -- (3.5,-.4);
        \node[above] at (2.25,-.25) {colori};

        \draw (0,-2.5) -- (7,-2.5);
        \draw (0,-2) -- (7,-2);
        \foreach \x in {0.5,1} {
          \draw (\x,-2.5) -- (\x,-2);
        };
        \node[above] at (0.75,-2.5) {$\rhd$};
        \node[above] at (3.75,-2.5) {$\dots$};
        \draw [decorate,
            decoration = {brace,amplitude=5pt}] (1,-1.9) -- (3,-1.9);
        \node[above] at (2,-1.75) {$Q$};
    \end{tikzpicture}
\end{center}
con $Q$ queue. Quindi, per la tesi di Church-Turing estesa, la complessità è $O(n^\alpha)$ per qualche $\alpha\in\mathbb{N}$. Si ha che
$$
    \text{Reachability} \in \text{P}
$$
Per quanto riguarda la \textbf{complessità in spazio}, si ha che i colori e $Q$ sono molto gradi, quindi
$$
    \text{Reachability} \in \text{PSPACE}
$$
\textbf{Esercizio}: migliorare questo risultato. In particolare, $\text{Reachability} \in \mathbb{L} = \text{SPACE}(\log n)$? $\text{Reachability} \in \text{SPACE}((\log n)^2)$?
\subparagraph{Modello nondeterministico} Si hanno diversi possibili stati futuri. Immaginiamo un grafo con un cammino da $u$ a $v$. Una macchina deterministica segue tutti i cammini uno ad uno, e per ognuno controlla se è quello corretto. Una macchina nondeterministica è in grado di ``indovinare'' il cammino corretto e di seguirlo.

Analizziamo la \textbf{complessità spaziale}. In questo caso non si ha bisogno dei colori. La macchina nondeterministica genera ad ogni passo un nuovo nodo nel working tape:
\begin{center}
    \begin{tikzpicture}
        \draw (0,1) -- (10.5,1);
        \draw (0,1.5) -- (10.5,1.5);
        \foreach \x in {0.5,1,2,2.5,3.5,4,5,5.5,6,7,7.5,8.5,9,10} {
          \draw (\x,1) -- (\x,1.5);
        };
        \node[above] at (0.75,1) {$\rhd$};
        \node[above] at (2.25,.9) {(};
        \node[above] at (3.75,1) {;};
        \node[above] at (5.25,.9) {)};
        \node[above] at (5.75,1) {;};
        \node[above] at (6.5,1) {$\dots$};
        \node[above] at (7.25,1) {;};
        \node[above] at (8.75,1) {;};
        \draw [decorate,
            decoration = {brace,amplitude=5pt}] (1,1.6) -- (2,1.6);
        \node[above] at (1.5,1.75) {$|V|$};
        \draw [decorate,
            decoration = {brace,amplitude=5pt}] (2.5,1.6) -- (3.5,1.6);
        \node[above] at (3,1.75) {$a$};
        \draw [decorate,
            decoration = {brace,amplitude=5pt}] (4,1.6) -- (5,1.6);
        \node[above] at (4.5,1.75) {$b$};
        \draw [decorate,
            decoration = {brace,amplitude=5pt}] (7.5,1.6) -- (8.5,1.6);
        \node[above] at (8,1.75) {$u$};
        \draw [decorate,
            decoration = {brace,amplitude=5pt}] (9,1.6) -- (10,1.6);
        \node[above] at (9.5,1.75) {$v$};
        \draw [decorate,
            decoration = {brace,mirror,amplitude=5pt}] (2,.9) -- (7,.9);
        \node[below] at (4.5,.75) {$\substack{\text{\normalsize archi}\\\text{\footnotesize worst case }O(|V|\cdot\log |V|)=n}$};
        \node[left] at (-.5,1.25) {input};

        \draw (0,-1) -- (10.5,-1);
        \draw (0,-.5) -- (10.5,-.5);
        \foreach \x in {0.5,1} {
          \draw (\x,-1) -- (\x,-.5);
        };
        \node[above] at (0.75,-1) {$\rhd$};
        \node[above] at (5,-1) {$\dots$};
        \node[ellipse,
            draw,
            minimum width=1.6cm,
            minimum height=.75cm] (A) at (1.75,-.75) { };
        \node[ellipse,
            draw,
            minimum width=1.6cm,
            minimum height=.75cm] (A) at (3.4,-.75) { };
        \node[below] at (1.75,-1.1) {$1.$};
        \node[below] at (3.4,-1.1) {$3.$};
        \node[left] at (-.3,-.75) {working t.};
    \end{tikzpicture}
\end{center}
\begin{enumerate}
    \item Genera il codice di un nodo (ad esempio, del nodo $a$)
    \item Controlla se esiste un arcp da $u$ ad $a$
    \begin{itemize}
        \item Se non esiste, ritorna ``no''
        \item Se esiste, vai avanti
    \end{itemize}
    \item Aggiungi un altro nodo (ad esempio, il nodo $b$)
    \item Controlla se esiste un arco da $a$ ad $b$
    \begin{itemize}
        \item Se non esiste, ritorna ``no''
        \item Se esiste, vai avanti
    \end{itemize}
    \item Sostituisci $a$ con $b$, e aggiungi un altro nodo (ad esempio, il nodo $c$)
    \item \dots
\end{enumerate}
\subparagraph{Esempio} Consideriamo il grafo
\begin{center}
    \begin{tikzpicture}[node distance={20mm}, main/.style = {}] 
        \node[main] (u) {$u$}; 
        \node[main] (a) [above right of=u] {$a$}; 
        \node[main] (b) [below right of=u] {$b$}; 
        \node[main] (c) [below left of=b] {$c$}; 
        \node[main] (d) [below right of=b] {$d$}; 
        \node[main] (e) [above right of=a] {$e$};
        \node[main] (f) [below right of=a] {$f$};
        \node[main] (v) [right of=f] {$v$}; 
        \draw[->] (u) -- (a);
        \draw[->] (a) -- (b);
        \draw[->] (b) -- (u);
        \draw[->] (d) -- (b);
        \draw[->] (b) -- (c);
        \draw[->] (c) -- (d);
        \draw[->] (a) -- (e);
        \draw[->] (a) -- (f);
        \draw[->] (f) -- (v);  
    \end{tikzpicture} 
\end{center}
Proviamo a trovare un cammino da $u$ a $v$:
\begin{itemize}
    \item Esiste un arco da $u$ ad $a$? Sì, quindi $ua$
    \item Esiste un arco da $a$ a $e$? Sì, quindi $uae$
    \item Esiste un arco da $e$ a $c$? No, quindi non esiste un cammino $uaec$
\end{itemize}
Esiste invece una computazione che genera il cammino $uafv$? Sì.
Si può notare come ci sia però un problema, ovvero i cicli (ad esempio $uabuabcdbcdb\dots$). In realtà, questo non è un problema: ragionando sulla complessità, e non sulla computabilità, consideriamo solo macchine di Turing che terminano.

Un modo per evitare i cicli è quello di immagazzinare in un working tape un contatore di passi eseguiti. Quando tale contatore raggiunge $|V|+1$, si può fermare la computazione e ritornare ``no''. 
$$
    \text{Reachability} \in \text{N}\mathbb{L} = \text{NSPACE}(\log n)
$$

\begin{definition}[Macchina Nondeterministica]
    Una macchina nondeterministica è una tupla $\mathcal{N}(K,\Sigma,\Delta,s)$, dove
    \begin{itemize}
        \item $K$ è un insieme finito di stati, di cui $s\in K$ è quello iniziale
        \item $\Sigma$ è un alfabeto finito, e $\rhd,\sqcup\in\Sigma$.
        \item $\Delta$ è la relazione di transizione, definita come 
        $$
        \Delta\subseteq K\times\Sigma^k\times (K\cup\{\text{yes},\text{no},\text{halt}\}) \times \Sigma^k \times \{\gets,\to,-\}^k
        $$
    \end{itemize}
    La relazione di transizione tra due configurazioni è definita come per le macchine deterministiche:
    $$
        (q,u_1,w_1,\dots,q_k,w_k) \to (q',u_1',w_1',\dots,q_k',w_k')
    $$
    con $(q',u_1',w_1',\dots,q_k',w_k')$ uno dei possibili risultati dell'applicazione di $\Delta$.
\end{definition}

\begin{definition}[Linguaggio deciso da una Macchina Nondeterministica]
    Un lin\-guag\-gio $L\subseteq(\Sigma\backslash\{\sqcup\})^*$ è deciso da una macchina nondeterministica $\mathcal{N}$ (con $\mathcal{N}(x)$ che termina sempre)
    \begin{itemize}
        \item[] se $\forall x\in L(\Sigma\backslash\{\sqcup\})^*$
        \begin{itemize}
            \item $x\in L$ $\Rightarrow$ esiste una computazione di $\mathcal{N}$ che inizia da $(s,\rhd,x,\rhd,\varepsilon,\dots,\rhd,\varepsilon)$ e termina in $(\text{yes},\dots)$
            \item $x\notin L$ $\Rightarrow$ tutte le computazioni di $\mathcal{N}$ che iniziano da $(s,\rhd,x,\rhd,\varepsilon,\dots,\rhd,\varepsilon)$ terminano in $(\text{no},\dots)$
        \end{itemize}
    \end{itemize}
\end{definition}
Se si ha una macchina che decide un linguaggio, bisogna definire la complessità spaziale e temporale su quella macchina. Per le macchine deterministiche, si ha che la complessità temporale è la lunghezza della computazione ne laso peggiore, per ogni possibile stringa di lunghezza $n$:
\begin{center}
    \begin{tikzpicture}[]
        \node (a1) at (0,0) {$\bullet$};
        \node (a2) at (0,-1) {$\bullet$};
        \node (a3) at (0,-2) {$\bullet$};
        \node (a4) at (0,-3) {$\bullet$};
        \node (b1) at (2,0) {$\bullet$};
        \node (b2) at (2,-1) {$\bullet$};
        \node (b3) at (2,-2) {$\bullet$};
        \node (b4) at (2,-3) {$\bullet$};
        \node (b5) at (2,-4) {$\bullet$};
        \node (b6) at (2,-5) {$\bullet$};
        \draw (a1) -- (a2) -- (a3) -- (a4);
        \draw (b1) -- (b2) -- (b3) -- (b4) --(b5) -- (b6);
        \node [above] at (a1) {$x\in L$};
        \node [left] at (a4) {yes};
        \node [above] at (b1) {$x\not\in L$};
        \node [right] at (b6) {no};
    \end{tikzpicture}
\end{center}
Ma nel caso nondeterministico si hanno degli alberi:
\begin{center}
    \begin{tikzpicture}[]
    \tikzstyle{level 1}=[sibling distance=4cm, level distance=1.2cm]
    \tikzstyle{level 2}=[sibling distance=1.6cm]
    \tikzstyle{level 3}=[sibling distance=1.6cm]
        \node (root) {$\bullet$}
        child {
            node (firstnode) {$\bullet$}
            child {
                node {no}        
                edge from parent 
            }
            child {
                node {$\bullet$}
                child {
                    node {$\bullet$}
                    child {
                        node {no}
                        edge from parent
                    }
                    child {
                        node {yes}
                        edge from parent
                    }
                    edge from parent
                }
                edge from parent 
            }        
            child {
                node {yes}
                edge from parent
            }
            edge from parent 
        }
        child {
            node {$\bullet$} 
            child {
                node {no}
                edge from parent
            }
            child {
                node {$\bullet$}
                child {
                    node {no}
                    edge from parent
                }
                child {
                    node {$\bullet$}
                    child {
                        node {no}
                        edge from parent
                    }
                    edge from parent
                }
                edge from parent         
            }
            edge from parent         
        }
        child {
            node {no}
            edge from parent
        };

    \node [above] at (root) {$x\in L$};
    \node [right] at (root.east) {$(s,\rhd,x,\dots)$};
    \node [left] at (firstnode) {$(q_1,w_1,\dots)$};
    \end{tikzpicture}
\end{center}

\begin{center}
    \begin{tikzpicture}[]
    \tikzstyle{level 1}=[sibling distance=4cm, level distance=1.2cm]
    \tikzstyle{level 2}=[sibling distance=1.6cm]
    \tikzstyle{level 3}=[sibling distance=1.6cm]
        \node (root) {$\bullet$}
        child {
            node (firstnode) {$\bullet$}
            child {
                node {no}        
                edge from parent 
            }
            child {
                node {$\bullet$}
                child {
                    node {$\bullet$}
                    child {
                        node {no}
                        edge from parent
                    }
                    child {
                        node {no}
                        edge from parent
                    }
                    edge from parent
                }
                edge from parent 
            }        
            child {
                node {no}
                edge from parent
            }
            edge from parent 
        }
        child {
            node {$\bullet$} 
            child {
                node {no}
                edge from parent
            }
            child {
                node {$\bullet$}
                child {
                    node {no}
                    edge from parent
                }
                child {
                    node {$\bullet$}
                    child {
                        node {no}
                        edge from parent
                    }
                    edge from parent
                }
                edge from parent         
            }
            edge from parent         
        }
        child {
            node {no}
            edge from parent
        };

    \node [above] at (root) {$x\not\in L$};
    \node [right] at (root.east) {$(s,\rhd,x,\dots)$};
    \node [left] at (firstnode) {$(q_1,w_1,\dots)$};
    \end{tikzpicture}
\end{center}
In questo caso, la complessità si può definire come l'altezza dell'albero.

\begin{definition}[Complessità Temporale di una Macchina Nondeterministica]
    U\-na\\macchina di Turing nondeterministica $\mathcal{N}$ con input $x$ richiede tempo $t$ se ogni possibile computazione di $\mathcal{N}$ su $x$ ha lunghezza al massimo $t$. 
    $\mathcal{N}$ richiede tempo $f(n)$ se, per ogni $x$, $\mathcal{N}$ termina su $x$ in tempo $f(|x|)$.
\end{definition}

\paragraph{Esempio} Immaginiamo che il grado del nondeterminismo di una macchina $\mathcal{N}$ sia 3, ovvero i nodi del suo albero hanno grado 3 ($d=3$). L'altezza è $f(|x|)$, e il numero di foglie è $d^{f(|x|)}$ (molto grande).


\begin{definition}[$\text{NTIME}(f(n))$]
    $L\in\text{NTIME}(f(n))$ se esiste una macchina di Turing nondeterministica $\mathcal{N}$ che decide $L$ in tempo $f(n)$.
\end{definition}
Notare che, ad esempio, $\text{TIME}(n^5)\neq\text{NTIME}(n^5)$. Quest'ultimo è l'insieme di tutti i linguaggi che possono essere decisi in tempo $n^5$ da una macchina di Turing nondeterministica.

\begin{proposition}
    $$
        \text{TIME}(f(n)) \subseteq \text{NTIME}(f(n))
    $$
\end{proposition}
Abbiamo che
$$
    \text{P} = \bigcup_{h\in\mathbb{N}} \text{TIME}(n^h)
    \qquad \qquad 
    \text{NP} = \bigcup_{h\in\mathbb{N}} \text{NTIME}(n^h)
$$
Quindi
$$
    \text{P} \subseteq \text{NP}
$$
Abbiamo definito la complessità temporale di una macchina nondeterministica. Possiamo definire anche la complessità spaziale come la configurazione (nodo dell'albero) massima.
\begin{definition}[Complessità Spaziale di una Macchina Nondeterministica]
    Una\\macchina di Turing nondeterministica con I/O $\mathcal{N}$ sull'input $x$ utilizza spazio $s$ se 
    $$
        s \geq \sum_{h=2}^{k-1} |w_h| + |u_h|
    $$
    Ovvero $s$ è il massimo su tutte le possibili computazioni di $\mathcal{N}$ su $x$
    $$
        s = \max\left( \sum |w_h| + |u_h| \right)
    $$
\end{definition}
$\mathcal{N}$ lavora in NSPACE$f(n)$, $\mathcal{N}$ decide $L$ in NSPACE$f(n)$.
\begin{align*}
    \text{PSPACE}=\mathbb{L} & ~\text{ spazio polinomiale su modello deterministico}\\
    \text{NPSPACE}=\text{N}\mathbb{L} & ~\text{ spazio polinomiale su modello nondeterministico} 
\end{align*}


\textcolor{Red}{TODO: finire lezione 13}


% LEZIONE 14
\textcolor{Red}{TODO: lezione 14}

% LEZIONE 15
\textcolor{Red}{TODO: lezione 15}

% LEZIONE 16
\textcolor{Red}{TODO: lezione 16}

\chapter{Riduzione e Completezza}
Capitolo 8 del libro. Alcuni problemi catturano la difficoltà di un'intera classe di complessità. La logica gioca un ruolo centrale in questo fenomeno.


\section{Riduzioni}
Si vuole risolvere un problema $A$ ``simile'' al problema $B$, e si possiede un algoritmo efficiente per $B$. Si possono eseguire una serie di trasformazioni:
$$
    \underset{\text{input per }A}{x} \to \underset{\text{input per }B}{x'} \to \text{algoritmo} \to \underset{\text{output per }B}{y'} \to \underset{\text{output per }A}{y}
$$
Se una trasformazione ha una complessità molto minore del problema, la complessità rimane la stessa. In particolare, la trasformazione più interessante è quella da $x$ a $x'$, e prende il nome di \textbf{riduzione}.
\begin{definition}[Riduzione]
    Una riduzione da $L_1$ a $L_2$ è
    $$
        R:\Sigma_1^*\to\Sigma_2^* \text{ ~ tale che ~ } x\in L_1 \Leftrightarrow R(x)\in L_2
    $$
    La complessità principale deriva dall'algoritmo per decidere $L_2$.
\end{definition}
$R$ dev'essere computabile in spazio logaritmico.

\paragraph{Esempio (Riduzione)} Si consideri il Graph 3-Coloring ($L_1$): dato un grafo $G=(V,E)$, decidere se è possibile definire $Col:V\to\{r,b,y\}$ tale che se $(u,v)\in E$ allora $Col(u)\neq Col(v)$.

Si consideri SAT ($L_2$): data una formula booleana, decidere se è soddisfacibile o meno. 
$$
    R:\text{Graph 3-Coloring}\to\text{SAT}
$$
ovvero
$$
    R(G) \text{ è soddisfacibile } \Leftrightarrow~ G \text{ è 3-colorabile}
$$
Abbiamo che 
\begin{eqnarray*}
    R(G) &=& \bigwedge_{u\in V} [(r_u\lor b_u\lor y_u) \land 
    (r_u\to \lnot b_u\land \lnot y_u) \land
    (b_u\to \lnot r_u\land \lnot y_u) \land
    (y_u\to \lnot r_u\land \lnot b_u)] \land\\
    & & \bigwedge_{(u,v)\in E} [(r_u\to \lnot r_v) \land (b_u\to \lnot b_v) \land (y_u\to \lnot y_v)]
\end{eqnarray*}
Se in una macchina di Turing con I/O si ha $G$ sul nastro di input e $R(G)$ sul nastro di output, la riduzione deve utilizzare spazio logaritmico sui working tape.

\begin{definition}[$L_1\leq L_2$]
    $L_1$ può essere ridotto a $L_2$ ($L_1\leq L_2$) se esiste una riduzione $R$ da $L_1$ a $L_2$.
\end{definition}

\begin{property} ($\circ$ è la composizione di funzioni)
    \begin{eqnarray*}
        &R_1 \text{ riduzione da } L_1 \text{ a } L_2&\\
        &R_2 \text{ riduzione da } L_2 \text{ a } L_3&\\
        &\Downarrow&\\
        &R_2\circ R_1 \text{ riduzione da } L_1 \text{ a } L_3&
    \end{eqnarray*}
\end{property}
Questo significa che $L_1\leq L_2$ e $L_2\leq L_3$ implica $L_1\leq L_3$. Inoltre, $L_1\leq L_2$ significa che, in termini di complessità, $L_1$ è al più difficile di $L_2$.

\begin{definition}[Chiusura per Riduzione]
    Sia $\mathcal{C}$ una classe di complessità. $\mathcal{C}$ è chiusa per riduzione se
    $$
        L_1\leq L_2 \text{ e } L_2\in\mathcal{C} ~\Rightarrow~ L_1\in\mathcal{C}
    $$
\end{definition}
Si può dimostrare che, ad esempio, P, NP, EXP, $\mathbb{L}$, N$\mathbb{L}$, PSPACE sono tutte chiuse per riduzione. \emph{Esercizio: trovare un esempio di una classe di complessità non chiusa per riduzione.}


\section{Completezza}

\begin{definition}[Completezza di una Classe di Complessità]
    Un linguaggio $L$ è completo per una classe $\mathcal{C}$ se
    \begin{enumerate}
        \item $L\in\mathcal{C}$
        \item $\forall L'\in\mathcal{C}$, $L'\leq L$
    \end{enumerate}    
\end{definition}

\begin{center}
    \begin{tikzpicture}
        \draw (0,0) ellipse (2 and 1);
        \node at (-1.8,0.8) {\(\mathcal{C}\)};
    
        \node (l) at (0,0.3) {$\bullet$};
        \node[above right] at (l) {$L$};

        \node (a) at (-1,-.2) {$\bullet$};
        \node (b) at (1.2,-.5) {$\bullet$};
        \node (c) at (0.2,-.7) {$\bullet$};
        \node[text width=3.8cm,right] (d) at (2.5, -.5) {\footnotesize tutti gli altri linguaggi possono essere ridotti a $L$};
        \draw (a) -- (d);
        \draw (b) -- (d);
        \draw (c) -- (d);
    \end{tikzpicture}
\end{center}

\begin{property}
    Se $\mathcal{C}$ e $\mathcal{C}'$ sono classi di complessità 
    \begin{itemize}
        \item chiuse per riduzione, e
        \item $\mathcal{C}\subseteq\mathcal{C}'$, e 
        \item $L'$ è completo per $\mathcal{C}'$, e
        \item $L'\in\mathcal{C}$
    \end{itemize}
    allora $\mathcal{C}=\mathcal{C}'$.
\end{property}

\begin{center}
    \begin{tikzpicture}
        \draw (0,0) ellipse (3 and 2);
        \node at (-2.5,1.5) {\(\mathcal{C}'\)};
        \draw (-.3,-.5) ellipse (2 and 1);
        \node at (-1.6,.6) {\(\mathcal{C}\)};

        \node (lp) at (1.5,1) {$\bullet$};
        \node[above left] at (lp) {$L'$};
        \node[] (tlp) at (3.5, 1.5) {\footnotesize completo};
        \draw[black!30] (lp) -- (tlp);

        \node (l) at (0,-.5) {$\bullet$};
        \draw[->] (lp) -- (l);

        \node[text width=5cm, right] (tl) at (4, 0) {\footnotesize se possiamo dimostrare che $L'$ cade in $\mathcal{C}$, allora tutti gli altri linguaggi possono essere ridotti a $L'$ ($L\leq L', \forall L$), e quindi $\mathcal{C}= \mathcal{C}'$};
        \draw[black!30] (l) -- (tl);
    \end{tikzpicture}
\end{center}



% LEZIONE 20
\subsection{Problema P-Completo: Circuit value}
Una \textbf{formula booleana} è coposta da variabili booleane $x_1,\dots,x_n$, da costanti $0,1$, e da operatori $\land,\lor,\lnot$. Una \textbf{funzione booleana} è 
$$
    \varphi:\{0,1\}^m\to\{0,1\}
$$

\paragraph{Esempio} m=3
$$
    \begin{rcases*}
        \varphi(\overset{x_1}{0},\overset{x_2}{0},\overset{x_3}{0})=1\\
        \varphi(0,1,0)=1\\
        \dots\\
        \varphi(1,1,1)=0
    \end{rcases*} \text{ dominio } |\{0,1\}^3|=8
$$
Questa funzione booleana ha valore 1 solo in due casi, ed è quindi equivalente alla formula booleana
$$
    (\lnot x_1\land \lnot x_2\land \lnot x_3) \lor (\lnot x_1\land 2\land \lnot x_3)
$$
Formule booleane ed espressioni booleane hanno lo stesso potere espressivo.\medskip

I \textbf{circuiti booleani} (boolean circuits) sono equivalenti a formule booleane ed espressioni booleane. Un circuito è composto da \emph{gates}, ovvero nodi di un grafo. I gates sono di tre tipi: variabili, costanti e operazioni.

\begin{center}
    \begin{tikzpicture}
        \node[circle, draw] (x1) at (0,0) {$x_1$};
        \node[circle, draw] (x2) at (2,0) {$x_2$};
        \node[circle, draw] (xm) at (4,0) {$x_m$};
        \node[circle, draw] (and1) at (1,-1) {$\land$};
        \node[circle, draw] (and2) at (3,-1) {$\land$};
        \node[circle, draw] (or) at (2,-2) {$\lor$};

        \draw[->] (x1) -- (and1);
        \draw[->] (x2) -- (and1);
        \draw[->] (x2) -- (and2);
        \draw[->] (xm) -- (and2);
        \draw[->] (and1) -- (or);
        \draw[->] (and1) -- (1,-1.8);
        \draw[->] (and1) -- (0.2,-1.7);
        \draw[->] (and2) -- (or);
        \draw[->] (or) -- (2,-2.8);
    \end{tikzpicture}
\end{center}
Un circuito booleano è un grafo diretto aciclico nel quale un risultato può essere utilizzato più volte. Il nodo senza archi uscenti è l'output gate.

\paragraph{Circuit Value} Dato un circuito booleano $C$ senza gate con variabili, calcolare il valore di output di $C$ (o, equivalentemente, decidere se ha valore 1).\medskip

Ad esempio, il valore del seguente circuito è 1.
\begin{center}
    \begin{tikzpicture}
        \node[circle, draw] (x1) at (0,0) {$0$};
        \node[circle, draw] (x2) at (2,0) {$1$};
        \node[circle, draw] (or1) at (-1,-1) {$\lor$};
        \node[circle, draw] (and1) at (3,-1) {$\land$};
        \node[circle, draw] (not) at (2,-2) {$\lnot$};
        \node[circle, draw] (and2) at (1,-3) {$\land$};

        \draw[->] (x1) -- (or1);
        \draw[->] (x1) -- (and1);
        \draw[->] (x2) -- (or1);
        \draw[->] (x2) -- (and1);
        \draw[->] (or1) -- (and2);
        \draw[->] (and1) -- (not);
        \draw[->] (not) -- (and2);
    \end{tikzpicture}
\end{center}
Qual è la complessità di questo problema? Per risolverlo, si può utilizzare il topological sort, iniziando dall'output gate. Quindi Circuit Value $\in$ P.

\begin{theorem}[Cook-Levin (1)]
    Circuit Value è P-completo.
\end{theorem}

\paragraph{Dimostrazione} (pagg.~166 e seguenti) Dobbiamo dimostrare che
\begin{enumerate}
    \item Circuit Value $\in$ P $\quad \checkmark$
    \item $\forall L\in$ P, $L\leq$ Circuit Value. Si ha $L\subseteq \Sigma^*$, bisogna trovare una funzione 
    $$
        R : \Sigma^* \to \text{ circuito senza variabili t.c.~} R(x)=1 \Leftrightarrow x\in L ~\land~ R \text{ computabile in spazio } O(\log n)
    $$
    Idea: se $L\in P$, allora esiste una macchina di Turing deterministica $\mathcal{M}$ che decide $L$ in tempo $O(n^k)$.
    \begin{center}
        \begin{tikzpicture}
            \draw (0,1.5) -- (5,1.5);
            \draw (0,2) -- (5,2);
            \foreach \x in {0.5,1,1.5,3,3.5,4} {
              \draw (\x,1.5) -- (\x,2);
            }
            \node[above] at (0.75,1.5) {$\rhd$};
            \node[above] at (1.25,1.5) {$x_1$};
            \node[above] at (2.25,1.5) {\dots};
            \node[above] at (3.25,1.5) {$x_n$};
            \node[above] at (3.75,1.5) {$\sqcup$};
            \draw [decorate,
                decoration = {brace,amplitude=5pt}] (1,2.1) -- (3.5,2.1) node [midway,above=0.1] {$x$};

            \node at (1.5,1.15) {\vdots};
    
            \draw (0,.5) -- (5,.5);
            \draw (0,0) -- (5,0);
            \foreach \x in {0.5,1,1.5,2.5,3,4,4.5} {
              \draw (\x,0.5) -- (\x,0);
            }
            \node[above] at (0.75,0) {$\rhd$};
            \node[above] at (1.25,0) {$y_1$};
            \node[above] at (2,0) {\dots};
            \node[above] at (2.75,0.5) {$\substack{\text{\large $q$}\\\downarrow}$};
            \node[above] at (2.75,0) {$y_j$};
            \node[above] at (3.5,0) {\dots};
            \node[above] at (4.25,0) {$y_h$};
    
            \node at (1.5,-.5) {\vdots};

            \draw (0,-1.5) -- (5,-1.5);
            \draw (0,-1) -- (5,-1);
            \foreach \x in {0.5,1} {
              \draw (\x,-1.5) -- (\x,-1);
            }
            \node[above] at (0.75,-1.5) {$\rhd$};
            
            \node[left] at (-.2,0.25) {``foto'' intermedia $i$};
            \node[text width=4cm,right] at (8,0) {\footnotesize si vogliono ``scattare'' tante ``foto'' quante ne servono per vedere cosa succede nella computazione};
            \draw [decorate,
                decoration = {brace,amplitude=5pt}] (5.2,2) -- (5.2,-1.5) node [text width=2cm,midway,right=0.2] {almeno $n^k$ ``fotografie''};
            \draw [decorate,
                decoration = {brace,mirror,amplitude=5pt}] (1,-1.7) -- (4.5,-1.7) node [midway,below=0.2] {queste ``fotografie'' sono lunghe $n^k$};
        \end{tikzpicture}
    \end{center}
    Sia $T_\mathcal{M}(x)$ la tabella di computazione (matrice) $|x|^k\times |x|^k$ della macchina $\mathcal{M}$ sull'input $x$, con $|x|^k$ limite temporale. In questa tabella le righe sono time step, mentre le colonne sono posizioni nella stringa della macchina. La cella $T_\mathcal{M}(i,j)$ rappresenta il contenuto della posizione $j$ della stringa di $\mathcal{M}$ al time step $i$ (ovvero dopo $i$ passi della macchina). Il valore di $T_\mathcal{M}(i,j)$ dipende solo dai contenuti delle posizioni $j-1$, $j$ e $j+1$ della stringa al time step $i-1$.

    Si formano quindi delle località (\emph{locality}). Se si trova la stessa situazione in un'altra parte del nastro, si avrà lo stesso effetto (Fig.~\vref{fig:circuit}).
    % \begin{center}
    %     \begin{tikzpicture}
    %         \draw (0,1.5) -- (2.5,1.5);
    %         \draw (0,2) -- (2.5,2);
    %         \foreach \x in {0.5,1,1.5,2} {
    %           \draw (\x,1.5) -- (\x,2);
    %         }
    %         \node[above] at (0.75,1.5) {1};
    %         \node[above] at (1.25,1.5) {0};
    %         \node[above] at (1.75,1.5) {0};
    
    %         \draw (0,.5) -- (2.5,.5);
    %         \draw (0,1) -- (2.5,1);
    %         \foreach \x in {0.5,1,1.5,2} {
    %           \draw (\x,0.5) -- (\x,1);
    %         }
    %         \node[above] at (0.75,0.5) {1};
    %         \node[above] at (1.25,0.5) {1};
    %         \node[above] at (1.75,0.5) {0};
    %     \end{tikzpicture}
    % \end{center}
    \begin{figure}[htb]
        \centering
        \includegraphics[width=.6\textwidth]{circuit.png}
        \caption{Costruzione del circuito.}
        \label{fig:circuit}
    \end{figure}
    I ``piccoli circuiti'' $C$ dipendono da $\mathcal{M}$, mentre l'intero circuito $C_\mathcal{M}(x)$ dipende da $\mathcal{M}$ e da $x$. $C_\mathcal{M}(x)$ può essere computato in spazio logaritmico. In particolare, poiché gli indici $i,j$ variano da 0 a $n^k$, è necessario $O(\log n^k)$ spazio per rappresentarli. \hfill$\square$
\end{enumerate}

\begin{corollary}
    $$
        L \text{ è P-completo} \quad \Leftrightarrow \quad L \in\text{P ~e~ Circuit Value} \leq L
    $$
\end{corollary}


\subsection{Problema NP-Completo: Circuit SAT}
\paragraph{Circuit SAT} Dato un circuito booleano $C$ con variabili, decidere se esiste un'assegnamento di valori $0,1$ alle variabili che rende l'output del circuito 1.

\begin{theorem}[Cook-Levin (2)]
    Circuit SAT è NP-completo.
\end{theorem}

\paragraph{Dimostrazione} Come la precedente, utilizzando però una macchina di Turing nondeterministica $\mathcal{N}$. Inoltre, ad ogni passo si hanno più scelte: si deve specificare se la scelta è 0 o 1. Si può definire una tabella $T_\mathcal{N}(x,c)$, con $c$ sequenza di scelte. Come output si ottiene un circuito $C_\mathcal{N}(x)$ con variabili, le quali arrivano dalle scelte della macchina nondeterministica. \hfill$\square$



\chapter{Problemi Completi}


\section{Problemi NP-Completi}
Dimostrare i risultati di NP-completezza è un ingrediente importante della nostra metodologia per lo studio dei problemi computazionali. Ci concentreremo su problemi NP e NP-completi. 

\subsection{SAT}
Sappiamo che Circuit SAT (SAT di formule booleane) è NP-completo. Formule booleane in CNF (conjunctive normal form) sono formule scritte come congiunzioni di disgiunzioni. Sono della forma
$$
    c_1 \land ... \land c_h
$$
dove ogni $c_i$ è una clausola, o disgiunzione di letterali. 
$$
    c_i \equiv \ell_{i,1} \lor ... \lor \ell_{i,j}
$$
Un letterale $\ell_{i,k}$ è una variabile booleana $x$ o la sua negazione $\lnot x$. Ad esempio, la formula booleana
$$
    (x_1\lor x_2\lor x_3\lor\lnot x_4) \land
    (x_2\lor x_4\lor\lnot x_5) \land
    (\lnot x_1\lor\lnot x_3\lor x_5)
$$
è in CNF. Mentre le formule in DNF sono banalmente molto semplici, quelle in CNF sono più difficili, perché si deve scegliere un letterale da ogni clausola e renderlo vero. Decidere CNF è difficile tanto quanto decidere una normale formula.
$$
    \text{SAT} \in \text{NP-completo}
$$
Solitamente con SAT ci si riferisce alla soddisfacibilità di formule in CNF.

\paragraph{3-SAT} Data la formula $\varphi$ in CNF tale che in ogni clausola ci sono al più 3 letterali, decidere se $\varphi$ è soddisfacibile. 

\begin{theorem}[NP-Completezza di 3-SAT]
    3-SAT è NP-completo.
\end{theorem}
\paragraph{Dimostrazione} Dobbiamo dimostrare che 
\begin{enumerate}
    \item 3-SAT $\in$ NP. Esiste un algoritmo nondeterministico che indovina la valutazione corretta in tempo polinomiale. $\quad \checkmark$
    \item SAT $\leq$ 3-SAT. Una generica clausola può essere mappata in una clausola con meno letterali. Prendiamo come esempio la clausola con 5 letterali 
    $$
        (x_1 \lor x_2 \lor x_3 \lor x_4 \lor x_5)
    $$
    Definiamo $z=x_1\lor x_2$. Inoltre, $z\to x_1\lor x_2\equiv \lnot z\lor(x_1\lor x_2)$. Quindi
    $$
        (z \lor x_3 \lor x_4 \lor x_5) \land (\lnot z \lor x_1 \lor x_2)
    $$
    Applicando questa tecnica più volte, si ottiene una formula con più variabili, ma ogni clause ha al più 3 letterali. \hfill$\square$
\end{enumerate}


\subsection{Caratterizzazione di NP}

\begin{definition}[Decidibile Polinomialmente]
    Una relazione binaria $R\subseteq\Sigma^*\times\Sigma^*$ è decidibile polinomialmente (deterministicamente) se, dato $(x,y)$, si può decidere in tempo $n^k$ se $(x,y)\in R$.
\end{definition}
In altre parole, $R$ è decidibile polinomialmente se esiste un algoritmo deterministico che la decide in tempo polinomiale.
\begin{definition}[Bilanciata Polinomialmente]
    Una relazione binaria $R\subseteq\Sigma^*\times\Sigma^*$ è bilanciata polinomialmente se 
    $$
        \exists h \quad \forall(x,y)\in R \quad |y|\leq |x|^h
    $$
\end{definition}

\begin{theorem}
    $L\in NP$ $\Leftrightarrow$ esiste $R$ decidibile polinomialmente e bilanciata polinomialmente t.c.~$L=\{x|\exists y~(x,y)\in R\}$
\end{theorem}
In altre parole, $y$ è un testimone, un certificato che $x\in L$. Con il certificato, si può controllare in tempo polinomiale deterministico se una soluzione è corretta.

% LEZIONE 21

\paragraph{Dimostrazione} 
\begin{itemize}
    \item[$\Rightarrow$] Se $L\in$ NP allora esiste una macchina di Turing nondeterministica $\mathcal{N}$ che decide $L$ in tempo $O(n^k)$. $\forall x\in L$, $\exists(c_1,c_2,\dots,c_{n^k})$ tali che $\mathcal{N}(x)=$ yes (ovvero insieme di scelte che produce in out\-put yes).
    $$
        R = \{ (x,y) ~|~ y=(c_1,\dots,c_{n^k}) \text{ t.c.~} \mathcal{N}^y(x)=\text{yes} \}
    $$
    con $R$ relazione binaria, $x$ stringa, $y$ stringa di interi che indica le scelte fatte. Vale che 
    $$
        L = \{ x ~|~ \exists y~(x,y)\in R \}
    $$
    poiché $\mathcal{N}$ è una macchina nondeterministica che decide $L$. La lunghezza di $y$ dev'essere polinomiale rispetto alla lunghezza di $x$ (bilanciata polinomialmente). $R$ è bilanciata polinomialmente in quanto $\mathcal{N}$ lavora in tempo polinomiale. $R$ è decidibile polinomialmente utilizzando $\mathcal{N}$ con le scelte definite da $y$.
    \item[$\Leftarrow$] Dato $x$,
    \begin{itemize}
        \item indovina $y$ di lunghezza al massimo $|x|^k$ (nondeterministico polinomiale sulla lunghezza di $x$)
        \item controlla $(x,y)\in R ~\Leftrightarrow$ tempo polinomiale 
        $$
            (|x|+|y|)^h \leq (|x| + |x|^k)^h
        $$
        (deterministico polinomiale sulla lunghezza di $x$)
    \end{itemize}
    Questi due assieme sono nondeterministici polinomiali per $L$. Invece di risolvibili efficientemente, questi problemi sono verificabili efficientemente. \hfill$\square$ 
\end{itemize}

\paragraph{NP-completezza di un linguaggio} In generale, per dimostrare che un linguaggio $L$ è NP-completo, si deve dimostrare che 
\begin{enumerate}
    \item $L\in$ NP
    \item dato un linguaggio $L''\in$ NP-completo, $\forall L''\in$ NP, $L''\leq L'$, $L'\leq L$. 
\end{enumerate}


\subsection{Independent Set}
\paragraph{Independent Set (IS)} Dato un grafo $G=(V,E)$ e un intero $k$, decidere se esiste  in $G$ un insieme di nodi $I$ tale che $|I|\geq k$ e $\forall u,v\in I$, $(u,v)\notin E$.\medskip

Dimostriamo che IS è NP-completo.
\begin{enumerate}
    \item Dati $G$ e $I$, in tempo polinomiale si può controllare $|I|\geq k$ e $\forall u,v\in I$, $(u,v)\notin E$. Quindi IS $\in$ NP.
    \item Si vuole dimostrare 3-SAT $\leq$ IS. Sia $\varphi$ la formula
    $$
        c_1\land c_2\land\dots\land c_k \equiv
        (\ell_{1,1}\lor\ell_{1,2}\lor\ell_{1,3}) \land
        (\ell_{2,1}\lor\ell_{2,2}\lor\ell_{2,3}) \land
        \dots \land
        (\ell_{k,1}\lor\ell_{k,2}\lor\ell_{k,3})
    $$
    Ad ogni letterale nella formula corrisponde un nodo nel grafo
    $$
        V = \{ \ell_{i,j} ~|~ \ell{i,j}\in\varphi \}
    $$
    I nodi sono esattamente i letterali nella formula.
    \begin{center}
        \begin{tikzpicture}
            \node[] (A) at (0,0) {$\bullet$};
            \node[] (B) at (2,0) {$\bullet$};
            \node[] (C) at (1,1) {$\bullet$};
            \node[below left] at (A) {$\ell_{h,1}$};
            \node[below right] at (B) {$\ell_{h,3}$};
            \node[above] at (C) {$\ell_{h,2}$};
            
            \draw (A) -- (B);
            \draw (B) -- (C);
            \draw (C) -- (A);
        \end{tikzpicture}
    \end{center}
    Ogni letterale è connesso alla sua negazione.
    \begin{center}
        \begin{tikzpicture}
            \node[] (A) at (0,0) {$\bullet$};
            \node[] (B) at (2,0) {$\bullet$};
            \node[] (C) at (1,1) {$\bullet$};
            \node[left=0.1] at (A) {$\lnot x$};
            \draw (A) -- (B);
            \draw (B) -- (C);
            \draw (C) -- (A);

            \node[] (D) at (-1,-2) {$\bullet$};
            \node[] (E) at (1,-2) {$\bullet$};
            \node[] (F) at (0,-1) {$\bullet$};
            \node[left=0.1] at (F) {$x$};
            \draw (D) -- (E);
            \draw (E) -- (F);
            \draw (F) -- (D);

            \node[] (G) at (3,-2) {$\bullet$};
            \node[] (H) at (5,-2) {$\bullet$};
            \node[] (I) at (4,-1) {$\bullet$};
            \node[below=0.1] at (G) {$\lnot x$};
            \draw (G) -- (H);
            \draw (H) -- (I);
            \draw (I) -- (G);

            \draw (A) -- (F);
            \draw (F) -- (G);
        \end{tikzpicture}
    \end{center}
    Controllare se $G$ ha un IS di dimensione $k$ (con $k$ numero di clausole della formula) è equivalente a controllare se $\varphi$ è soddisfacibile. 
    \begin{center}
        \begin{tikzpicture}
            \draw (0,1.5) -- (5,1.5);
            \draw (0,2) -- (5,2);
            \foreach \x in {0.5,1} {
              \draw (\x,1.5) -- (\x,2);
            }
            \node[above] at (0.75,1.5) {$\rhd$};
            \draw [decorate,
                decoration = {brace,amplitude=5pt}] (1,2.1) -- (3,2.1) node [midway,above=0.1] {$y$};

            \node at (2.5,1.1) {\vdots};
    
            \draw (0,0) -- (5,0);
            \draw (0,.5) -- (5,.5);
            \foreach \x in {0.5,1} {
              \draw (\x,0.5) -- (\x,0);
            }
            \node[above] at (0.75,0) {$\rhd$};
            \draw [decorate,
                decoration = {mirror,brace,amplitude=5pt}] (1,-.1) -- (3.45,-.1) node [midway,below=0.1] {$G$};
            \draw [decorate,
                decoration = {mirror,brace,amplitude=5pt}] (3.55,-.1) -- (4.5,-.1) node [midway,below=0.1] {$k$};
    
            \node[right] at (5.2,1.75) {input};
            \node[right] at (5.2,1) {working tapes};
            \node[right] at (5.2,0.25) {output};            
        \end{tikzpicture}
    \end{center}
    \hfill$\square$
\end{enumerate}


\subsection{Clique}

\paragraph{Clique (Cricca)} Dato un grafo $G=(V,E)$ e un intero $k$, decidere se esiste un insieme di nodi $C$ tale che $|C|\geq k$ e $\forall u,v\in C$, $(u,v)\in E$.\medskip

\begin{enumerate}
    \item Dati $G$ e $C$, in tempo polinomiale si può controllare $|C|\geq k$ e $\forall u,v\in C$, $(u,v)\in E$. Quindi Clique $\in$ NP.
    \item Si vuole dimostrare IS $\leq$ Clique. Per IS, siano $G$ il grafo e $k$ l'intero. Per Clique si costruisce un grafo $G'=(V,E')$ e $E'=\{(u,v)~|~(u,v)\notin E\}$. Esiste un independent set sse esiste una clique. \hfill$\square$
\end{enumerate}
Come esempio per il punto 2, si considerino il grafo $G$ e il suo complemento $G'$
\begin{center}
    \begin{tikzpicture}
        \node at (3,2) {$G$};
        \node[] (A) at (0,0) {$\bullet$};
        \node[] (B) at (2,0) {$\bullet$};
        \node[] (C) at (1,1) {$\bullet$};
        \node[below left=0.1] at (A) {$a$};
        \node[below right=0.1] at (B) {$b$};
        \node[above=0.1] at (C) {$c$};
        \draw (A) -- (B);
        \draw (B) -- (C);
        \draw (C) -- (A);

        \node[] (D) at (4,0) {$\bullet$};
        \node[] (E) at (6,0) {$\bullet$};
        \node[] (F) at (5,1) {$\bullet$};
        \node[below left=0.1] at (D) {$d$};
        \node[below right=0.1] at (E) {$e$};
        \node[above=0.1] at (F) {$f$};
        \draw (D) -- (E);
        \draw (E) -- (F);
        \draw (F) -- (D);

        \draw (C) -- (D);
        %%%%%%%%%%

        \node at (10,2.5) {$G'$};

        \node[] (A') at (9,2) {$\bullet$};
        \node[] (B') at (9,.5) {$\bullet$};
        \node[] (C') at (9,-1) {$\bullet$};
        \node[] (D') at (11,2) {$\bullet$};
        \node[] (E') at (11,.5) {$\bullet$};
        \node[] (F') at (11,-1) {$\bullet$};
        \node[left=0.1] at (A') {$a'$};
        \node[left=0.1] at (B') {$b'$};
        \node[left=0.1] at (C') {$c'$};
        \node[right=0.1] at (D') {$d'$};
        \node[right=0.1] at (E') {$e'$};
        \node[right=0.1] at (F') {$f'$};
        \draw (A') -- (D');
        \draw (A') -- (E');
        \draw (A') -- (F');
        \draw (B') -- (D');
        \draw (B') -- (E');
        \draw (B') -- (F');
        \draw (C') -- (E');
        \draw (C') -- (F');
    \end{tikzpicture}
\end{center}


\subsection{Isomorfismo tra Sottografi}
\paragraph{Subgraph Isomorphism (SGI)} Dati $G_1,G_2$, decidere se esiste un sottografo di $G_1$ che è isomorfo a $G_2$.\medskip 

Ad esempio, $G_2$ è isomorfo a un sottografo in $G_1$, ma $G_2'$ non è presente in $G_1$.
\begin{center}
    \begin{tikzpicture}
        \node at (1,1.6) {$G_1$};
        \node (A) at (0,0) {$\bullet$};
        \node (B) at (2,0) {$\bullet$};
        \node (C) at (1,1) {$\bullet$};
        \node (D) at (2,-1.5) {$\bullet$};
        \node (E) at (0,-1.5) {$\bullet$};
        \draw (A) -- (B);
        \draw (B) -- (C);
        \draw (C) -- (A);
        \draw (A) -- (E);
        \draw (B) -- (D);
        \draw (E) -- (D);

        \node at (5,1.6) {$G_2$};
        \node (F) at (4,0) {$\bullet$};
        \node (G) at (6,0) {$\bullet$};
        \node (H) at (5,1) {$\bullet$};
        \draw (F) -- (G);
        \draw (G) -- (H);
        \draw (H) -- (F);

        \node at (9,1.6) {$G_2'$};
        \node (I) at (8,0) {$\bullet$};
        \node (J) at (10,0) {$\bullet$};
        \node (K) at (9,1) {$\bullet$};
        \node (L) at (9,-1) {$\bullet$};
        \draw (I) -- (J);
        \draw (J) -- (K);
        \draw (K) -- (I);
        \draw (I) -- (L);
        \draw (J) -- (L);
    \end{tikzpicture}
\end{center}
Clique $\leq$ SGI: siano $G$, $k$ il grafo e l'intero per Clique. Per SGI si costruiscono i grafi $G_1=G$ e $G_2=G_k$ (grafo completo con $k$ nodi). Esiste una clique di dimensione $k$ in $G$ sse esiste un sottografo isomorfo a $G_k$ in $G$. Ad esempio, cercare una 3-clique in $G$ significa cercare un pattern del tipo 
\begin{center}
    \begin{tikzpicture}
        \node (F) at (4,0) {$\bullet$};
        \node (G) at (6,0) {$\bullet$};
        \node (H) at (5,1) {$\bullet$};
        \draw (F) -- (G);
        \draw (G) -- (H);
        \draw (H) -- (F);
    \end{tikzpicture}
\end{center}



\subsection{Isomorfismo tra Grafi}
\paragraph{Graph Isomorphism (GI)} Dati $G_1,G_2$, decidere se $G_1$ è isomorfo a $G_2$.\medskip

Sappiamo che GI $\in$ NP, ma non si sa se è NP-completo. Se si dimostrasse che P $\neq$ NP, allora $\exists L$, $L\in$ NP$\backslash$P $\land$ $L\not\in$ NP-completo.



\subsection{Programmazione Intera}
\paragraph{Integer Programming (IP)} Dato un insiele di disuguaglianze lineari, decidere se esiste un assegnamento di interi alle variabili che soddisfa tutte le disuguaglianze.\medskip

3-SAT $\leq$ IP: sia $\varphi=(\ell_{1,1},\ell_{1,2},\ell_{1,3})\land(\dots)$ e ogni letterale $\ell_{i,k}$ è del tipo $p,q,r$. Per ogni variabile nella formula, consideriamo $x_p,x_{\lnot p}$. Consideriamo i seguenti vincoli per tutte le variabili nella formula
$$
\begin{cases*}
    0\leq x_p\leq 1 \\
    0\leq x_{\lnot p}\leq 1 \\
    x_p+x_{\lnot p}\geq 1 \\
    x_p+x_{\lnot p}\leq 1 \\
    \vdots \\
    x_p+x_{\lnot q}+x_r\geq 1 \\
\end{cases*}
$$
IP $\in$ NP-completo. La programmazione lineare (su $\mathbb{R}$) $\in$ P.



% LEZIONE 22
\subsection{Knapsack}
\paragraph{Knapsack} Datio $S$ insieme di oggetti, $i\in S$, con peso $w_i$ e valore $v_i$, e una sacca di peso $W$, massimizzare $V$.\medskip

Knapsack $\in$ NP-completo.

\begin{theorem}
    Ogni istanza di Knapsack può essere risolta in tempo $O(|S|W)$.
\end{theorem}
Questo è un algoritmo pseudo-polinomiale, in quanto dipende da $W$. Se fissiamo o limitiamo $W$, allora l'algoritmo è polinomiale. Tutti i problemi per i quali non esistono algoritmi pseudo-polinomiali sono detti fortemente NP-completi.



%%%%%%%%%%
\section{Problemi co-NP-Completi}

\begin{center}
    \begin{tikzpicture}
        \draw (0,0) ellipse (2cm and 1cm);
        \draw (3,0) ellipse (2cm and 1cm);
        \draw (-1.2,.8) -- (-1.2,-.8);
        \draw (4.2,.8) -- (4.2,-.8);

        \node[left] (npc) at (-2.5,0) {NP-completo};
        \node at (-.2,0) {NP};
        \node at (1.5,0) {P};
        \node at (3.2,0) {co-NP};
        \node[right] (cnpc) at (5.5,0) {co-NP-completo};
        \draw (npc) -- (-1.8,0);
        \draw (cnpc) -- (4.8,0);
    \end{tikzpicture}
\end{center}

\begin{theorem}
    $$
        L \text{ è completo per } \mathcal{C}
        \quad \Leftrightarrow \quad
        \overline{L} \text{ è completo per co-} \mathcal{C}
    $$
\end{theorem}
\paragraph{Dimostrazione} Se $L$ è completo per $\mathcal{C}$, significa che $L\in\mathcal{C}$ e $\forall L'\in\mathcal{C}$, $L'\leq L$. Quindi $\overline{L}\in$ co-$\mathcal{C}$ (definizione). Sia $L''\in$ co-N$\mathcal{C}$, allora $L''=\overline{L'}$, con $L'\in\mathcal{C}$. Sappiamo che $L'\leq L$. $L'\subseteq(\Sigma')^*$, $L\subseteq(\Sigma)^*$.
$$
    \exists R : (\Sigma')^*\to(\Sigma)^* \qquad R(x)\in L \Leftrightarrow x\in L'
$$
con $R$ computabile in spazio logaritmico. Vogliamo dimostrare che $L''\leq\overline{L}$. Si consideri $R$
$$
    x\in L'' ~\Leftrightarrow~ 
    x\not\in L' ~\Leftrightarrow~
    R(x)\not\in L ~\Leftrightarrow~
    R(x)\in\overline{L}
$$
$R$ è computabile in spazio logaritmico, quindi $L''\leq\overline{L}$. \hfill$\square$\medskip 

Ad esempio, SAT ($\forall v\dots$) è NP-completo $\Rightarrow$ UnSAT ($\exists v\dots$) è co-NP-completo.


\subsection{Validity}
\paragraph{Validity} Data una formula $\varphi$, decidere se è valida. $\varphi$ è valida sse $\lnot\varphi$ è insoddisfacibile.\medskip 

La riduzione da Validity a UnSAT si ottiene mettendo $\neg$ davanti a $\varphi$. 





%%%%%%%%%%
\section{Problemi $\mathbb{L}$-Completi}
Poiché le nostre riduzioni sono in spazio logaritmico, è difficile decidere se un linguaggio è o meno in $\mathbb{L}$. 

\begin{theorem}
    $\forall L\in\mathbb{L}$, $L\neq\emptyset$, e $L\neq\Sigma^*$, allora $L$ è $\mathbb{L}$-completo.
\end{theorem}

\paragraph{Dimostrazione} Sia $L\in\mathbb{L}$, $L\neq\emptyset$ ($x\in L$), e $L\neq\Sigma^*$ ($y\not\in L$). Sia $L'\in\mathbb{L}$
$$
    \forall x'\in L' ~ R(x')=x 
    \qquad 
    \forall x'\not\in L' ~ R(x')=y
$$
Si può dimostrare che questa riduzione è computabile in spazio logaritmico? Sì, perché $L\in\mathbb{L}$. In spa\-zio logaritmico si può decidere se $x'\in L'$, e si ottiene o $x$ o $y$.\hfill$\square$




%%%%%%%%%%
\section{Problemi N$\mathbb{L}$-Completi}

\subsection{Reachability}
Vedi il reachability method nella sezione \vref{sec:reachability}. Reachability $\in$ N$\mathbb{L}$-completo. $\forall L\in$ N$\mathbb{L}$, $\mathcal{N}$ decide $L$, $G_\mathcal{N}(x)$. Se $L\in$ N$\mathbb{L}$, allora $G_\mathcal{N}(x)$ può essere computato in spazio logaritmico.


\subsection{2-SAT}
2-SAT $\in$ N$\mathbb{L}$-completo. Abbiamo una congiunzione di clausole $c_1\land c_2\land\dots\land c_k$, dove ogni clausola è del tipo
$$
    c_i \equiv (\ell_{i,1}\lor\ell_{i,2}) \equiv (\lnot\ell_{i,1}\to\ell_{i,2}) \equiv (\lnot\ell_{i,2}\to\ell_{i,1})
$$
Queste implicazioni logiche possono essere viste come archi di un grafo.

\paragraph{Esempio} $\varphi=(x\lor\neg y)\land(\neg x\lor z)\land(\neg z\lor\neg y)$

\begin{center}
    \begin{tikzpicture}
        \node (x) at (0,.5) {$x$};
        \node (nx) at (0,-2) {$\neg x$};
        \node (y) at (1.5,0) {$y$};
        \node (ny) at (1.5,-1.5) {$\neg y$};
        \node (z) at (-1.5,0) {$z$};
        \node (nz) at (-1.5,-1.5) {$\neg z$};

        \draw[->] (x) -- (z);
        \draw[->] (nx) -- (ny);
        \draw[->] (y) -- (x);
        \draw[->] (y) -- (nz);
        \draw[->] (z) -- (ny);
        \draw[->] (nz) -- (nx);
    \end{tikzpicture}
\end{center}
In questo caso $y$ raggiunge $y'$. Se, invece, prendiamo come esempio la formula 
$$
    \varphi'=(x\lor\neg y)\land(\neg x\lor z)\land(\neg z\lor\neg y)\land(y\land\neg z)\land(z\land y)
$$
\begin{center}
    \begin{tikzpicture}
        \node (x) at (0,.5) {$x$};
        \node (nx) at (0,-2) {$\neg x$};
        \node (y) at (1.5,0) {$y$};
        \node (ny) at (1.5,-1.5) {$\neg y$};
        \node (z) at (-1.5,0) {$z$};
        \node (nz) at (-1.5,-1.5) {$\neg z$};

        \draw[->] (x) -- (z);
        \draw[->] (nx) -- (ny);
        \draw[->] (y) -- (x);
        \draw[<->] (ny) -- (z);
        \draw[<->] (ny) -- (nz);
        \draw[->] (z) -- (y);
        \draw[->] (nz) -- (nx);
        \draw[<->] (nz) -- (y);
    \end{tikzpicture}
\end{center}
Ora sia $y\to\neg y$ che $\neg y\to y$: $\varphi'$ è insoddisfacibile.





%%%%%%%%%%
% \section{Gerarchia Polinomiale}


% \textcolor{Red}{TODO: finire lezione 22}
%%%%%%%%%%


%%%%%%%%%% 
\newpage
\renewcommand{\listtheoremname}{Lista di Teoremi}
% \begin{multicols}{2}
% {
% \footnotesize
\listoftheorems[ignoreall,show={theorem,
% corollary,
% lemma,
% property,
% proposition
}]
% }
% \end{multicols}

\newpage
\renewcommand{\listtheoremname}{Lista di Definizioni}
% \begin{multicols}{2}
% {
% \footnotesize
\listoftheorems[ignoreall,show=definition]
% }
% \end{multicols}
%%%%%%%%%%


\end{document}

